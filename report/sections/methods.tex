\chapter{Methods\label{Methods}}

This chapter deals with the design of our system, an overview of the
implementation and technical aspects on the data that we work with. We
will be taking a look at how the first prototype was designed and how
it branched out to several different versions due to experiments with
various heuristics applied to the data. We will go through the
individual modules of the system and how they communicate through the
data formats. After giving a thorough description on the data we work
with in the system, we finally round off the chapter with some of the
techincal conclusions that we got from building the system.

\section{System overview and design\label{SystemOverviewAndDesign}}

In this section, we lay out the design of the prototype system. In
3.1.1 we describe how the first prototype is put together while, in
3.1.2, we describe the different ways we have tried to structure the
system. Finally, in 3.1.3, we mention how the applied filters have led
to several different versions of the matrices that make up the heart
of the system.

\subsection{The first prototype}

We have made a modular design that is divided into five major
components representing our system. The first module is a webcrawler
called \textit{DiseaseCrawler}. It gathers preliminary data from
Rarediseases.info and saves it in a specified data format on the disk
\ref{Crawler}, allowing the next module to read it in and process
it. \textit{The Search and Information Retrieval} (SIR) module
searches and retrieves MedLine records from Pubmed.org in accordance
to the data gathered in the crawler module. SIR saves its data in a
file under the disease name, containing up till 500 MedLine records
per disease-search and a description if one is found \ref{SIR}. This
allows the third module \textit{TermDoc} to read in the data, convert
all the disease to sub term document matrices and construct a large
term document matrix from the data from the sub matrices, saving the
term document matrix in Matrix Market format \ref{MatrixMarket}. As an
option one can be include a filter just before the construction of the
large term document matrix (or on the sub matrices), e.g. a stemmer
and/or stop word remover or other kinds of filters, like
log-transformation of the term counts and the much used term frequency
- inverse document frequency (TF-IDF). The resulting data from the
TermDoc module can now be used for querying. This is done through the
\textit{QueryInterface} module that implements the cosine measure to
perform correlation between vectors, i.e. between a query vector and a
document vector in our term document matrix. To be able to give a
disease name instead of just a MedLine record, we need to score the
disease in relation to the given query vector. This is done by a
consensus method where we select the number of MedLine records to use
a basis of the consensus so that each MedLine record has one or more
disease names attached to it (the same MedLine record can be returned
from different diseases). We then sum up the score from each MedLine
record under the given disease names and sort the total scores. Those
with the highest total score has the greatest correlation with the
search query and therefore these are the most likely to be our
disease.

The modular design of the prototype system allows the different
modules to be replaced with more efficient modules or modules that
gathers data from different sources - as long as the new module
conforms to the specified data formats. It allows for an easy addition
of new heuristics and filtering modules to specific points of the data
modelling. (data modelling?!? data flow? data ???)

\subsection{Branching the prototype\label{DiseaseMatrix}}

A major branch in our design sprung when we realised the potential of building a disease matrix instead of a term document. Though severely simplifying our data, this model is a reduced version of the term document matrix and is a lot easier and faster to work with. It provides a testing environment and, as we shall see later (\fxnote{REF: til histogram-sammenlignings-tests af label og term-doc}), the disease matrix does not score that far from the more detailed term document matrix.

The disease matrix is based on the same data as the term document matrix. It differs in the way that it contains all the information that we have about each disease from the MedLine records, summed into one vector describing the disease. The new matrix still needs to contain the same terms as the term document matrix but is now made of a disease vectors instead of document vectors. This makes the individual disease vector less sparse than a document vector. The same preprocessing options, as applied to the term document matrix, can be applied to the disease matrix just as easily.

\subsection{Filtering for new matrices}

Adding filters to the prototype system spawns new matrices to work on
(as shown on in the figure below). The sub matrices generated from the
MedLine records are generated both as stemmed on non-stemmed. From
these, two large term document matrices are generated and two
different disease matrices. Adding another heuristic, the large
matrices are TF-IDF transformed (with and without normalization).

Yet another pool of sub matrices are created by SVD and dimensionality
reduction, leading to yet another large disease matrix to perform
tests and potential keyword extractions on.

Again we mention that filters and heuristics should be pretty easy to
add since only the data formats in between the modules needs to be
kept in order.

<OVERALL-DIAGRAM goes here!>

\section{Modules}

In this section, we will go through the individual modules described
in \ref{SystemOverviewAndDesign}. We will be giving an
overview of the module, its parts and the way that the data, in
between the modules, is structured. We will also be looking at the
filter modules, the auxiliary modules and the modules used for data
analysis.

\subsection{Crawler\label{Crawler}}

Overview and Purpose

The crawler is the first step in creating a database of MedLine
records, containing information about rare diseases, since its main
purpose is to gather information about what to search for in PubMed.

As described in \ref{Rarediseases_info}, Raredisease.info contains a
list of rare diseases and a varying degree of information on each
specific disease. We were referenced to the website by Dr. Henrik
Jørgensen \fxnote{REF to the dude} and a crawler, to collect
information from the site, was the first module to be made. The
crawler goes through every disease from A to Z,0-49 and saves the name
of the disease and (if any exist) the synonyms, the specialized search
string for PubMed and a description of the disease. This information
is then used by a S.I.R.-module \footnote{Search and Information
  Retrieval} which is described in the following chapter (\ref{SIR}).

Our main module is named \textit{DiseaseCrawler}. It crawls the Rarediseases.info webpage and gathers information as described above. It is a rather large module since it has to take a series of anomalies into account when crawling Rarediseases.info (like disfunctional subpages, strange characters and unexpected white-spaces). It utilizes the auxiliary modules \textit{TextCleaner} and \textit{IOmodule}, described in \ref{AuxModules}

Method and data describtion

The crawler accepts a list of letter for which to gather information from, e.g. ['A'] means gather all diseases beginning with A. It utilizes the html parser library Beautifulsoup \fxnote{REF: skal lige nævnes/beskrives et sted} to parse the webpages, looking for disease names, synonyms, handcrafted searches and uids. The crawled data is all stored in database of dictionaries on the form:

{'terms': '', 'desc': '', 'db': u'omim', 'syn': [u'Pectus excavatum', u' macrocephaly and dysplastic nails', u'Familial short stature', u' developmental delay', u' pectus abnormalities', u' distinctive facies', u' and dysplastic nails'], 'uid': u'600399'}

The name of the file, that each of the datasets are stored in, is the name of the disease. The above shows a example for the disease 'Zori Stalker Williams syndrome'. 'terms' will contain the handcrafted search string for PubMed if one is found while 'desc' contains the description of the disease if one is available (unfortunately on Rarediseases.info only 6.94\% contains one). 'db' represents the choice of database to use and refers to either PubMed or OMIM in our current cases. It is needed to know how to search for the disease in Entrez, as described in the following section. 'syn' is a list containing the various synonyms associated with the disease and used for finding more relevant information about the disease. 'uid' is a unique identifier found on all OMIM links and in some cases on pre-calculated pubmed searches.

\subsection{SIR (Search and Information Retrieval)\label{SIR}}

Overview and Purpose

The SIR\footnote{Search and Information Retrieval}-module reads in the information saved by the crawler (or any other crawler). It uses Entrez for accessing, searching and retrieving MedLine records from PubMed. The information contained within the MedLine record represents our knowledge base about the diseases. The SIR-module is set to search in such a way that we hope to optimize getting records that are actually relevant to the given disease. A maximum of 500 Medline records are downloaded per disease using a two phase search, all containing an abstract.

The main module is called \textit{PubmedSIR} and is used to search and retrieve MedLine records from the PubMed database.

As mentioned above the searching is split into two phases where it looks for at most 250 MedLine records in each phase. It first examines whether there is a handcrafted search string to search for or whether the disease has a PubMed or OMIM unique-id (uid). When searching for the handcrafted search string, PubmedSIR automatically adds the additional search options 'AND hasabstract[text]' to the string. This makes sure that all the MedLine records, that are returned, contains an abstract. This is unfortunately not possible when dealing with PubMed/OMIM uids which means that we have to employ other means to ensure that the returned MedLine records contains an abstract. This is done by the method \textit{getMedLineList} which takes a list of PMIDs, downloads them from PubMed and runs through all the MedLine records, selecting only those containing an abstract. Making a local cleaning to ensure that the MedLine records contain abstracts also means that we can not guarantee 500 MedLine abstracts for a disease even thought they are available. This is a minor fault in our system that should have been corrected if time allowed it but we have chosen to continue with the information we have available. Alternatively additional search options could be to also include constraints for getting only abstracts in english, only records published after a certain date etc. For options see \cite{PubmedHelpSearch}

PubmedSIR relies primarily on the function \textit{getArticleIDsFromMultipleSources} for searching across the two major databases of Entrez - PubMed and OMIM. We have chosen not to remove duplicate MedLine records between the first and second phase of the search because it is our belief that if a record is present in both searches, the terms is worth counting twice. The searches are done as the described below.

First phase of the search:

\begin{enumerate}

\item 1a) Search for term if it is present, OR

\item 1b) Search for pubmed/omim uid.

\item If we have obtained less than 250 MedLine records,

\item 2) Search for the disease name on pubmed.

\item 3) Eliminate duplicates.

\end{enumerate}

Second phase of the search:

\begin{itemize}

\item 1) Calculate all possible combinations of the synonyms.

\item 2) Search for the combined synonym. If a combination returns 0 results then eliminate all future searches that contains this combination since pubmed put in 'AND' between search terms ( meaning that future searches containing this combination will also return 0 results).

\item 3) Fill up until we get at most 500 MedLine records.

\end{itemize}

Method and data description

The primary function of the module is \textit{gatherOfAllThings},
which reads in the information that were saved by the crawler. This
information is passed onto \textit{getArticleIDs} that in turn calls
\textit{getArticleIDsFromMultiSource} which searches the items
specified within the disease dictionary. getArticleIDs is also the
function that keeps track of the number of MedLine records that are
downloaded for each disease.

A typical dictionary read in from the crawler looks as follows:


{'disease x': {'syn' : [xx, yy, zz], 'term' : string, 'uid' : string,
    'description' : string, 'db' : pubmed|omim}, 'disease y': {'syn' :
    [aa, bb], 'term' : string, 'uid' : string, 'description' : string,
    'db' : pubmed|omim}, ...}

gatherOfAllThings completes by performing a writeout of the MedLine
records to the disk in the following format:

{'disease a': [pmid1, pmid2, pmid3...], 'disease b' : [pmidx,
    pmidy,...], ...}

The SIR module uses the following auxillary modules \ref{AuxModules}:

\textit{SearchTermCombiner} which is a simple module that is used to
combine search terms in all of its possible unique combinations. This
code has been found at \cite{} and is slightly
modified. \fxnote{webpage for searchtermcombiner code}

\textit{IOmodule} handles Input/Output.

\textit{TextCleaner} is used to sanitize the input strings.

For more information about the gathered dataset, see \ref{Database}

\subsection{TermDoc\label{TermDoc}}

Overview and purpose

The information gathered from the SIR-module now needs to be processed
to allow queries to be made on it. An often used method in Information
Retrieval (IR) is the vector space model \ref{VectorSpace} that
represents the gathered information as document vectors (in a term
document matrix). The result is that queries to the system can be made
using a query vector, getting a similarity score/measure against all
documents contained within the model. In the following, we will go
through the creation of the sub term document matrices, the large term
document matrix and the compressed disease matrix.

We use a two-phase approach to construct the complete term document
matrix.

In the first phase, we make a sub term document matrix for each
disease containing the information from the MedLine records. We split
up the abstract, title and MeSH terms if present. Various filters can
here be applied to the terms, e.g. stemming and stop word removal. We
choose to remove any kind of punctuation and the like because
otherwise the terms remain very noisy ("blood" and "blood." would be
two different terms). We keep single letters (except for 'a' which
counts as a stop word), because many diseases contains single letters
as identification of which type they are, e.g. 'Hemoglobin C disease'.

The second phase simply goes through the sub term document matrices
and fill the term count values into complete term document matrix.

There are two main modules. The first, called \textit{TermDoc}, is
able to make sub term document matrices from a folder containing
MedLine records and to combine a folder containing sub term document
matrices into a complete term document matrix. The second one is
called \textit{LabelMatrix} and makes a matrix with disease vectors
instead of document vectors.

Method and data describtion

The main function for creating the sub term document matrices from a
folder containing MedLine records is
\textit{medlineDir2MatrixDir}. This function requires a hash table
containing hashes for all the terms and pmids of the MedLine
record. The need for hash tables comes from the fact that the data
structures, we have chosen, does not support string entries. So hashes
can be made by the \textit{createTermAndPmidHashes}. This function
goes through a folder containing MedLine records, while building a
term and pmid hash table. When medlineDir2Matrix has read in the hash,
it proceeds by calling \textit{gatherMatrixData} on each file within
the MedLine record folder. gatherMatrixData extracts information from
the file by the use of the auxillary module \textit{RecordHandler}
\ref{AuxModules}. The information can be specified by the user -
title, abstract and MeSH terms are chosen by default, as these seem to
give a good overall description of a disease. This is also the place
to perform stop word removal and stemming. We have chosen to create
both a stemmed and an unstemmed matrix in order to test what performs
best. medlineDir2Matrix then calls \textit{populateMatrix} with the
data from gatherMatrixData. This creates and returns a term document
matrix. Last it calls IOmodule \ref{AuxModules} to write the created
term document matrix to the disk in Matrix Market format
\ref{MatrixMarket}.

For creation of the large term document matrix, the function
\textit{reateTermDoc} is used. This goes through the folder containing
the sub term matrices and places the term count for each of the
MedLine records in the right place in the term document matrix. This
is basically done by looking up in the hash table for were to place
them. If the same MedLine record exists in two different diseases, the
term counts are summed. When done, it is written to the disk in Matrix
Market format.

The disease matrix is created by calling \textit{constructLabelMatrix}
(label is what we have given to call a disease) with a folder of sub
term document matrices as input. It then runs through every of the sub
term document matrices and calls \textit{getColumnSum} for each of
them. This sums the sub matrices to a single vector and returns one
row for each of the diseases which can be used to represent
it. getColumnSum has the option of making the average column sum
instead of just summing them. This option can be used to normalize the
disease vectors, should it be needed. The disease matrix is, like the
term document matrix, based on hash tables. These can be created by
running createDiseaseHash on a folder containing sub term document
matrices.

The TermDoc module uses the following auxillary modules
\ref{AuxModules}:

RecordHandler, which is used for extraction with the records contained
within the MedLine records, e.g. 'AB' for abstract etc.

\textit{FilterInterface} used to get access to Porter stemmer and stop
word removal of string.

IOmodule and TextCleaner as mentioned in the previous section.

\fxnote{ordentlig-syg STATSTIK}

\subsection{FilterInterface and heuristics}

Overview and purpose

When dealing with the amounts of information, in a system like this,
there is a need to make some modifications to the data. We choose to
sanitize the input information to our system by removal of
punctuations, commas, etc. and by making every term lowercase. This
helps reduce the number of different terms in the system. This has the
side-effect that it also removes punctuations within describtion of
e.g. chromosome errors. Taking an example, the string "1q42.4-qter
duplication" will be split into '1q42', '4', 'qter' 'duplication'. We
do, however, not consider this to be a problem since the query
recieves the same preprocessing as the term document matrix and it
should still be possible to retrieve the right
information\footnote{Using regular expression, it is possible to
  preserve the above string as: '1q42.4-qter', 'duplication' but we do
  not believe it important for the prototype}. The simple string
cleaning also allows the user to use other notations for the same
gene\footnote{'1q42.4-qter' and 1q42-4-qter amounts to the same}.

Another common technique in IR is to use stop word removal. This is
because words like 'this', 'the' and 'a' are very common and thereby
do not contain any information in the term-independent vector space
model. In some circumstances it is also normal to remove single letter
characters but as some diseases are characterized by having a special
type (as mentioned in \ref{TermDoc}), we choose not to remove single
letters. However, our stop word remover unfortunately does remove 'a'
due to its frequency in the english language.

\fxnote{ordentlig-syg STATSTIK}

Some numbers about filtering:

Making a 'raw' term document matrix, without any filtering results in
1,945,966 terms.

After sanitizing the information there are 465,220 terms.

And And after stemming there is a further reduction to 390,766 terms.

There are a couple of modules involved with filtering. We have made a
FilterInterface module to provide easy access to the different
filters.

FilterInterface

This is simply a gateway to various filters that are implemented in
separate modules. It is designed to return e.g. a stemmer or a stop
word remover that can be run on the abstracts before the term document
matrix is constructed. In the current prototype, it contains the
modules \textit{StopwordRemover}, \textit{Stemmer} and
\textit{TFIDFMatrix}.

StopwordRemover

The stop word remover allows for list of stop words to be supplied by
the user. By default it uses the nltk.corpus.stopwords of english stop
words which contains 127 stop words. There are other languages present
in the stop word corpus for a total of 2431 words, e.g. german,
danish, swedish norwegian and others. We do not know if any important
words are removed due in a multi language stop word removal, and have
therefore chosen not to remove anything but english stop words. We
assume that most of our information is in english and have chosen only
to remove english stop words. It is possible to setup additional
options within the SIR \ref{SIR} so that it will only gather MedLine
records containing abstracts in english but this is preserved for a
later version of the system.

Stemmer

To preserve flexibility our system allows another stemmer to be sent
to the function replacing the default stemmer. The default stemmer is
\textit{nltk.PorterStemmer().stem} that performs stemming on our
abstract, title and MeSH terms to "smooth" out the terms. It is only
advisable to run the stemmer after the stop word remover. This is
mainly because the stemmer changes some stop words so that they will
not be recognised by the stop word remover, e.g. performing stemming
on 'this' results in 'thi' which is not included in the default stop
word corpus.

TFIDFMatrix

The TFIDFMatrix module is used to perform the TF-IDF transformation of
a term document matrix using the equation from \ref{TFIDF}. It
performs the transformation by reading the term frequency (tf) from an
original matrix only containing term counts and then by making a
log-transformation of the tf. For finding the inverse document
frequency (IDF), we have made a precalculated hash table containing
the number of documents that the different terms are present in such
that $idf = \frac{number of document}{number of document term occurs
  in}$. We then store the calculation of $tf \cdot idf$ at the terms
position within the term document matrix. The transformed term
document matrix is then saved to the disk. The module then performs
normalization of the document vector to make sure that each document
has the same influence on the result of a query (used for the
enhancing the speed of the cosine similarity calculation
\ref{VectorSimilarity}). The normalization is done as usual vector
normalization $\frac{\overrightarrow{a}}{||a||}$.

\subsubsection{Auxillary modules\label{AuxModules}}

Auxillary modules are used by the different modules to perform tasks
like input/output, stemming, stop word removing, cleaning text string
or combining synonyms into search queries.

IOmodule

Performs various I/O function. For instance when a module is writing
or reading objects like hash tables to/from the disk, it simply calls
the \textit{pickleIn} or \textit{pickleOut} function with a path. The
object is then written or read. It is also able to return a sorted
list of file references from a folder which is very useful when one
needs to keep track on how far the process has come. This module also
allows for term document matrices to be written or read from the disk
using the \textit{Matrix Market} format \ref{MatrixMarket}.

TextCleaner

This module performs string manipulation like removing tags from html
code, sanitizing strings for punctuation, commas and all other special
characters, decoding various html characters. Most of these task are
obtained by return a regular expression for the specific task.

RecordHandler

The RecordHandler module is used to read information fields from
MedLine records which it returns as a dictionary containing the
requested fields.

\subsection{SearchInterface}

The search interface implements different approaches of measuring
similarity/distance between the query vector and document vector in
our term document matrix. Our two choices of measure in the vector
space model is the cosine similarity measure and a simple sum
measure. Instead of going through all the rows (documents) in our
matrix, we take the terms from the query vector and look up the only
the documents containing one or more of the queried terms. This limits
our search space and significantly enhances the time it takes to
process a query.

SearchInterface

This is the simple search interface that allows the different search
methods to be called, hence acting as a gateway like the
FilterInterface described above.

CosineMeasure

This module is used to perform a search using the cosine measure for
distance calculations between the query vector and the rows of our
term document matrix. It uses \textit{SearchTermDoc} to get the row
indices of which rows the query terms are present in. It then sums up
the scores in accordance to the occurrence of the query term. This
should resemble usual cosine measuring between vector when performed
on a pre-normalized vector space \ref{VectorSimilarity}.

SumMeasure

The SumMeasure module is used to perform a different kind of
measure. It performs a summing of the entries in the in the document
vector according to the terms of the query vector. It basically acts
as the cosine measure but is used on a vector space that is not
normalized. Again note that the reason we can compare the two measures
is because of the simplifications made in \ref{VectorSimilarity}.

SearchTermDoc

This module is used as a support module for performing searches. Given
a search vector, it will return the row indices of the term document
vectors that contain any of the terms. It can extract the term columns
with the relevant documents indices\footnote{Documents containing the
  query term} and it can create the hashes needed for normalization
and for column element counts. It is also performs reverse look up of
pmids (documents) given a pmidhash value.

\subsection{Data analysis tools}

In order keep track of the amount of information that we have
collected, we have made a crude module for gathering information. It
can be used to get the total number of pmids including duplicates, the
number of MedLine records containing a title or the number of diseases
that contains a description. In addition to this, we made a module to
perform hierarchical clustering of the diseases of top 20 results
returned by our system.

The modules, that are part of the data analysis suite, is
\textit{Cluster} which performs the clustering,
\textit{DistanceMeasure} which implements different
distance/similarity measures to be used within Cluster and
\textit{Stat} which is able to count various information fields
contained within the MedLine records.

Cluster

The Cluster module contains various functions in relation to
hierarchical clustering and drawing of dendrograms of the returned
clusters. The hierarchical clusterings has unfortunately not been made
as generic as it could be. For now, slight modifications are required
between running either on a disease matrix / term document matrix or a
sub term document matrix. We have no intentions of performing a full
clustering on a term document matrix, as it simply contains too many
entries to consider clustering -at least with the resources available
currently. The hierarchical clustering and dendrogram functionalities
are based on \cite{CollectiveIntelligence} with slight modification to
adopt it for our data.

DistanceMeasure

This module simply implements its own cosine measure functions for
sparse and dense matrices.

Stat

This module is able to count the various different fields within the
MedLine records, e.g. how many have a title, a MeSH, etc. It is also
able to count how many duplicate pmids there is.

\section{The database\label{Database}}

Our raw dataset consists mainly of two parts that are gathered
independently. The first part is the information gathered from
Rarediseases.info. The files reside in a subfolder called
\textit{rarediseases\_info} containing 6881 text files. Looking into
one of these files it is possible to see exactly what information have
been used to retrieve the medline records for a specific disease. The
second part of our raw dataset is the information gathered from PubMed
by the SIR module. This information can be find in the subfolder
\textit{medline\_records}. Again we have chosen to store it as plain
text files which enable the use of GNU unix/linux command line tools
for quick looks inside or using grep to look for specific words inside
a disease file.

Due to the limit on 500 abstracts per disease and with a total of
6,881 different rare diseases from Raredisease.info, the theoretical
upper limit on the number of abstract is 3,440,500. But since the
diseases are rare and the crawled information from Rarediseases.info
faulty a times, in reality the number of returned MedLine records is
much smaller. In fact, we only have 602,466 unique MedLine records
(about 2.8 million from the theoretical limit) and approximately
1,036,432 when counting duplicates. One of the MedLine records is even
shared among 240 diseases which indicates that it is an overview over
many diseases. There are also 505 diseases that do not return any
information at all. This means the remaining 6,376 diseases, on
average, have 94.49 MedLine records each. When searching PubMed, we
need to impose the 500-limit on the number of abstracts because (even
though the diseases are rare) some of them will return a lot of
information. Kidney cancer, though on the list of rare diseases, will
return 51,393 hits (\date{January 3, 2010}) with a search on PubMed
(only those with an abstract) and this is without considering any
synonyms or possibly handcrafted search terms.

We have choosen to remove the 505 empty disease entries from our
dataset because, without any information about these diseases, our
system will be unable to find/diagnose them.

More statistics on the data

Out of the 1,036,432 MedLine records, 1,036,417 has title. This is nearly 100\% (99.99\%). Not all of the MedLine records have MeSH terms although 924,026 has. This is 89.15\% of all the entries. This might be a bit misleading as it count the duplicate ones too, try to get count without duplicates.

\section{Techincal conclusions}

When performing text mining, a robust is needed to be able to handle
various situations. This became apparent to us after having written
the first version of it. Due to the inconsistency of
Rarediseases.info, it crashed every time that it ran into a new
special case on Rarediseases.info. Therefore, when crawling website
based on incomplete topics like rare diseases, its important to make
proper error handling and logging which diseases were missed in the
first run since errors are near certainty. As a sidenote on this, the
BeautifulSoup module is a really useful tool when crawling html since
it is able to correct and prettify many common website errors.

Gathering data from PubMed was performed by the Entrez module which on
several occasions crashed. This gave birth to the need to gather the
MedLine records in chunks to be able to resume them at any point. When
collecting data from OMIM- or PubMed-uid links, there is no way to
ensure that the returned MedLine records contains abstract and this
needs to be dealt with locally. Before performing text or data mining,
ALWAYS seek the permission of those running the site or sites. During
this project, we got banned from Rarediseases.info once and from
Orpha.net twice. This was not because we broke any laws or rules but
because most websites today protects themselves from harmful bots,
replay attacks and other risks to the website. If you do not have
permission to find information on the website, at least make sure you
give your credentials, browser type, etc. with the crawler.

Constructing a term document matrix requires a sparse data structure
for being stored on disk and in memory but when working with document
vectors, making them dense can mean a huge speed up on arithmetic
operations. Choosing to rewrite some of the more computational parts
to a low level language like C would also increase performance. Saving
intermediate steps along the way while making term document matrices
also allows other preprocessing steps to be performed, if needed later
in the project and is recommendable for large projects.

When querying the matrices, it is important to try several different
methods since the most well known or obvious one, might not be the
best choice (as we shall see in the following chapter).
