\chapter{Changes from original synopsis}

The project has changed title from \textit{``Support decision system
  for diagnosing rare diseases using machine learning for medical text
  mining''} to \texttt{``Support decision system for diagnosing rare
  diseases using vector space model and medical text mining''} this
was because the original

The original abstract has also been changed, the projects focus is no
longer on clustering, but ranking diseases amongst a top 20 given a
list of sympsoms. The reason for this is that the physicians would
like a system that gave them disease suggestions instead of organized
and structured information, which was the orignal though when
clustering was mentioned. The result is that the system

The system original success criterias has been changed from
\begin{enumerate}
\item The ability to correctly diagnose 50 rare diseases by search
  terms pro- vided by doctor Henrik L. Jørgensen. A successful hit is
  the correct diagnosis being among the top 20 search results.
\item The use of as few references (clicks) as possible to narrow in
  the search.  This should be 5 or less.
\item The use of as few search terms as possible. This should be 5 or
  less depending on the rarity of the disease.
\item How quickly the physician can locate the correct article
  compared to when using PubMed.
\end{enumerate}

This has been changed because we did not recieve the 50 golden
standard diseases on which to test our system. We recieved 5 test
cases on which we performed a blind test. The lack of test cases made
us look into alternatives which we found in BMJ, and some randomly
chosen strings characterizing symptoms from Orpha.net.

The system is to deliver a suggestion list of top 20 diseases given a
symptom list, therefore is does not make sense to talk about using as
few references as possible.

The use of as few symptoms as possible, is partially this in affect. We have misinterpreted few a whole lot. In table \ref{BlindTestDiseaseName} the symptoms we received 
