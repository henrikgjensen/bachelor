\chapter*{Changes from Original Synopsis}

The project has changed title from \textit{``Support decision system
  for diagnosing rare diseases using machine learning for medical text
  mining''} to \texttt{``Support decision system for diagnosing rare
  diseases using vector space model and \\ medical text
  mining''}. This was because the original project was focused on
gathering and organizing information using clustering algorithms, to
help physicians locate the right PubMed article to identify a
disease. The new focus was to providing a list of 20 suggestions for
possible diseases given a list of symptoms, this was best carried out using
information retrieval techniques such as the vector space model.

The system original success criteria were
\begin{enumerate}
\item The ability to correctly diagnose 50 rare diseases by search
  terms provided by chief physician Henrik L. J\o rgensen. A
  successful hit is the correct diagnosis being among the top 20
  search results.
\item The use of as few references (clicks) as possible to narrow in
  the search. This should be 5 or fewer.
\item The use of as few search terms as possible. This should be 5 or
  fewer depending on the rarity of the disease.
\item How quickly the physician can locate the correct article
  compared to when using PubMed.
\end{enumerate}

Reasons for changing the success critera
\begin{itemize}
\item[Ad 1.] This has been changed because we did not receive the 50 golden
standard diseases on which to test our system. We received 5 test
cases upon which we performed a blind test. The lack of test cases made
us look into alternatives which we found in BMJ, and some randomly
chosen strings characterizing symptoms from Orpha.net.

\item[Ad 2.] The system is to deliver a suggestion list of top 20 diseases given a
symptom list, therefore is does not make sense to talk about using as
few references as possible.

\item[Ad 3.] The use of as few symptoms as possible, is partially still in effect,
though we have misinterpreted the word few. The symptoms lists we
received for blind test from Henrik L. J\o rgensen can be seen in
table \ref{BlindTestDiseaseName}. These are comprehensive. So the
phrase ``few symptoms'' is relative to the observer.

\item[Ad 4.] Our system never reached a state where it was possible for the
physicians to test it, therefore the last of the original success
criteria was not met.
\end{itemize}

Therefore only a single success criteria has been specified for the revised
edition, and that is the system's ability to correctly place rare
diseases amongst top 20, given a symptom list, which will be the main
focus of the project.
