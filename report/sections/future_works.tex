\chapter{Future Works\label{FutureWorks}}

In this chapter we present some of the ideas for improvements that could be implemented into the system.

\subsubsection{Statistical model}
Gathering information about prior probabilities of the diseases could
help produce better term weights for the term document matrix. The
more information that could be found the more specific the
model. Gathering information about age, gender, racial and location
dependencies of the diseases should be able to improve the system
ability to correctly find the right disease.

Improved classifications of the diseases could help in clustering the
diseases to gain a better overview of them.

 ... \fxnote{Do not know what to write}

\subsubsection{Construction of other term document matrices}
A term document matrix based on keywords extracted by LSA could prove
most useful.

\subsubsection{Improvement on queries}
The possibility of comparing to queries to the system, could be useful
to a phycisian. He could then compare the results from two slightly
different queries, and then inspect the returned results.

Addition query options like preference for author, year, journal
etc. Could make it easier for a phycisian already knowing a bit about
what he is about to find.

Term mapping would be able to increase the recall of the system, this
could help retrieve the right information. Mapping synonyms to
eachother and searching for these.

Boolean searches, allowing the user that terms should be present and
other should not.

Visual representation of disease cluster, including keyswords
extracted by LSA.

A visual representation of the results something like table
\ref{FutureGUI}. Were ``Keywords'' are extracted those that best
describe the disease, given the information we have about
it. ``Score'' is the calculated score that the system has given the
query. And ``PMID'' is the list of PMID's that the score is derived
from. It should be possible to expand the PMID's in different level to
see title, then abstract and perhaps full informtion about the MedLine
record.
\begin{table}
  \begin{tabular}{|r|r|r|r|}
    \hline
    Disease name & Keywords & Score & PMIDs \\
    \hline
    Disease$_1$ & "Cancer, lever, blood, aortic ..." & 0.8242 & $18134923$, \dots, $3289472$ \\
    \hline
    {} & {} & {} & {} \\
    \hline
    Disease$_20$ & "Cancer, lever, blood, aortic ..." & 0.2412 & $17584932$, \dots, $16191394$ \\
    \hline
  \end{tabular}
  \label{FutureGUI}
\end{table}

\subsubsection{Improvements regarding text / data retrieval}
Locate more sources to gather information from, Orpha.net would be an ideal candidate. During the project we did also come across a rare diseases webpage from the Swedish Governments health care section, the page were unfortunenately down for maintanence

\begin{itemize}
  \item Improvements involving queries / searches
  \begin{itemize}
    \item Possibility of querying sub a clusters of returned diseases
    \item Improve the recall in the system by performing term mapping. Meaning a search term will be mapped to all its synonyms thereby hopefully improving the recall of information.
  \end{itemize}
  \item Visualization / GUI
  \begin{itemize}
    \item Visual representation of disease cluster, including keywords around a search query.
    
  \end{itemize}
  \item Improvement regarding the text mining / data retrieval for the system
  \begin{itemize}
    \item Incorporate multi disease database, e.g. mining from Orpha.net, and other sites like it. (Ask permission first)
    \item Implement a threshold for the minimum amount of information a disease should include to be part of the special gang.
    \item A golden standard for each disease, thereby applying \fxnote{REF Poultry} classification algorithm to PubMed for each disease and thereby getting the most relevant Medline records for each disease. This could hopefully improve results greatly. Though acquiring a golden standard is resource intensive task, so this is probably not very likely to happen any day soon.
    \item Implementin outlier detection: Examining how outlier removal affect the search result could be valuable. This process might concentrate the remaining information, thereby making it more likely to get the right information back. Therefore, it might be advantageous to perform a clustering on the abstracts and select some criteria on which to remove outlying abstracts. The criteria could be either removal based on some distance (threshold) to the cluster average or to remove some percentage of those abstract that lies farthest away from the cluster center. The latter of these seems to be the most fair one - mainly because it removes abstracts in proportion to the size of the cluster. One fact remains when it comes to outlier detection; care needs to be taken when removing outliers since one might just remove that single piece of correct information needed to identify the right disease.
  \end{itemize}
   \item Redesign of the system, start using a database for storing and retrieving information, optimizing by rewriting ineffective code to C code. 
  
\end{itemize}







