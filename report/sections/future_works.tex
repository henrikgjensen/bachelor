\chapter{Future Works\label{FutureWorks}}

In this chapter, we present some ideas for improvements that 
could be implemented into the system.

\subsubsection{Statistical Model}
Gathering information about prior probabilities of the diseases could
help produce better term weights for the term document matrix. The
more information that could be found, the more specific the
model. Also, gathering information about age, gender, racial and location
dependencies of the diseases should be able to improve the systems
ability to correctly find the correct disease. Further, it would present
the physician with extra filtering options for a query.

Improved classification of the diseases. To begin with, one could use 
the MeSH term hierachy for a prior clustering of the diseases and --- for
better classification --- remove outlier documents from each disease.

The outlier removal process might concentrate the remaining information, 
thereby making it more likely that the correct information is returned. 
Therefore, it might be advantageous to perform a clustering upon the
abstracts and select some criteria upon which to remove outlying abstracts.
The criteria could be either removal based on some distance (threshold) 
to the cluster average, or removal of some percentage of those abstracts
that lie farthest away from the cluster center. The latter seems to be 
the most fair option - mainly because it removes abstracts in proportion to 
the size of the cluster. One fact that remains when it comes to outlier 
detection; care needs to be taken when removing outliers, as one might just 
remove that single piece of correct information needed to identify the right disease.

It could also be possible to strengthen a classifier by gathering disease
domain keywords (e.g. coronary diseases), either from pre-existing lists or by text 
mining. The domain keywords have the strength that they can come from
more common and well-described diseases. They can all-in-all be described as
and expanded MeSH list to classify by.

Methods for keyword extraction --- like LSA --- should be tested.

\subsubsection{Construction of Other Term Document Matrices}
Using SVD (LSA) to reduce noise and construct a term document matrix
based on a semantic space.

Constructing a protein/gene name free matrix for a query based on general describing
terms/symptoms. This would be most useful if the physician could enable/disable this 
functionality

Redesign of the storage and retrieval system. A database for storing and retrieving 
information that is more efficient than the current I/O that the system use. Much
could probably be improved by writing certain data structures and methods in C - 
for instance the ones dealing with the large matrices.

\subsubsection{Improvements upon Queries}
The possibility of comparing two queries in the system could be useful
for a physician. He could then compare the results from two slightly
different queries and inspect the returned results. This functionality
could be implemented with a cosine measure that L$_2$ normalizes the query
vector.

Additional query options, like preferences for author, year, journal
etc. could make it easier for a physician who has a small amount of prior knowledge as to 
what he is about to find.

Term mapping would be able to increase the recall of the system and help retrieve the 
correct information. Term mapping is the idea of mapping synonyms to each other 
include these in a query. This approach would most likely work best on a disease 
matrix to avoid overview articles that, for instance, contain many synonyms.

Boolean searches, allowing the physician to determine that some terms should be present and
others should not. It could also be used to link certain terms together, like ``adrenal
insufficiency'' that is split in two in the current system, thereby ensuring either both
terms occur in a document or none does.

Visual representation of disease cluster, including keyswords extracted by LSA.

Visual representation of the results, perhaps like table
\ref{FutureGUI} below. Here those terms that best
describe the disease (``Keywords'') are presented to the physician to give him or her
and idea of what the disease is dealing with. ``Keyword'' could also be related terms
to the query terms themselves. ``Score'' is the calculated score that the system has given the
query, and ``PMID'' is the list of PMIDs that the score is derived
from. It should be possible to expand the PMIDs on different levels to
see title, the abstract and perhaps the full information about the MedLine
record.
\begin{table}
\begin{tiny}
  \begin{tabular}{|r|r|r|r|}
    \hline
    Disease name & Keywords & Score & PMIDs \\
    \hline
    Disease$_1$ & "Cancer, liver, blood, pressure ..." & 0.8242 & $18134923$, \dots, $3289472$ \\
    \hline
    {} & {} & {} & {} \\
    \hline
    Disease$_20$ & "Cancer, coronary, blood, aortic ..." & 0.2412 & $17584932$, \dots, $16191394$ \\
    \hline
  \end{tabular}
  \end{tiny}
  \label{FutureGUI}
\end{table}

\subsubsection{Improvements Regarding Text/Data Retrieval}
Locate more sources to gather information from. Orpha.net could be an ideal candidate. Furthermore,
during the project, we also came across a rare diseases webpage from the Swedish 
Government's health care system, but the page was unfortunately unaccessable due to maintanence.\fxnote{swedish homepage}

During this project, we also found that much of the data on the diseases were inadequate. Many diseases are only represented
by one or a few more abstracts. Though these moght be correct, they do not contain enough synonyms and decribing terms to
be hit by most queries. Therefore there should be a lower bound on the number of abstracts per disease.
