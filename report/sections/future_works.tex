\chapter{Future Works\label{FutureWorks}}

In this chapter we present some of the ideas for improvements that could be implemented into the system.

\subsubsection{Statistical model}
Gathering information about prior probabilities of the diseases could
help produce better term weights for the term document matrix. The
more information that could be found the more specific the
model. Gathering information about age, gender, racial and location
dependencies of the diseases should be able to improve the system
ability to correctly find the right disease.

Improved classifications of the diseases could help in clustering the
diseases to gain a better overview of them.

 ... \fxnote{Do not know what to write}

\subsubsection{Construction of other term document matrices}
A term document matrix based on keywords extracted by LSA could prove
most useful.

Making a term

\subsubsection{Improvement on queries}
The possibility of comparing to queries to the system, could be useful
to a phycisian. He would then be able compare the results from two
slightly different queries, look for key differences.

Term mapping would be able to increase the recall of the system, this
could help retrieve the right information. Mapping synonyms to
eachother and searching for these.

Visual representation of disease cluster, including keyswords
extracted by LSA.

\subsubsection{Graphical User Interface}
A visual representation of the results something like the table
\ref{FutureGUI}. Were ``Keywords'' are extracted those that best
describe the disease, given the information we have about
it. ``Score'' is the calculated score that the system has given the
query. And ``PMID'' is the list of PMID's that the score is derived
from. It should be possible to expand the PMID's in different level to
see title, then abstract and perhaps full informtion about the MedLine
record.
\begin{table}
  \begin{tabular}{|r|r|r|r|}
    \hline
    Disease name & Keywords & Score & PMIDs \\
    \hline
    Disease$_1$ & ``Cancer, lever, aortic \ldots'' & 0.8242 & $18134923$, \dots, $3289472$ \\
    \hline
    {} & {} & {} & {} \\
    \hline
    Disease$_2$ & ``Myeoblastic, fibrosis, oesteogenesis, \ldots'' & 0.2412 & $17584932$, \dots, $16191394$ \\
    \hline
  \end{tabular}
  \label{FutureGUI}
\end{table}

\subsubsection{Improvements regarding text / data retrieval}
Locate more sources to gather information from, Orpha.net would be an
ideal candidate. During the project we did also come across a rare
diseases web page from the Swedish Governments health care section,
the page were unfortunenately down for maintanence and only contained
247 diseases. But even a small contribution might be the determining
factor for placing a disease within top 20.

The need for minimum amount of information gathered per rare disease,
either as a minimum number of MedLine records attached to it or a
special describtion containing a near complete symptom list.


  \begin{itemize}
    \item Incorporate multi disease database, e.g. mining from Orpha.net, and other sites like it. (Ask permission first)
    \item Implement a threshold for the minimum amount of information a disease should include to be part of the special gang.
    
  \end{itemize}
   \item Redesign of the system, start using a database for storing and retrieving information, optimizing by rewriting ineffective code to C code. 
  
\end{itemize}







