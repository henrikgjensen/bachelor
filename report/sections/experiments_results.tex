\chapter{Experiments and Results\label{ExperimentsResults}}

In this section, we introduce the test cases we intend to use. We go
through the details of our cosine and sum measure scoring schemes that
we will be using to test our systems ability to rank a correct disease
given a list of symptoms. This is followed by various test results
using different similarity measures and different matrix models. By
comparing the individual top scores of each measure on a given matrix,
stemmed or non-stemmed, we find the most efficient measure to score
data on our system. It should here be noted that each time we score a
given disease, we do it by taking the top 3000\footnote{A number
  chosen at random in between our total number of different diseases}
of the documents returned by the similarity measure and the Search
module. \\

We then look at \fxnote{Skriv noget her} ... Note: cluster and
semantic section are still to be written. \\

Finally, we take discussion on the potential noise of overview
articles and test the potential of concensus normalization.

\section{Test cases}

\subsubsection{BMJ}
In order to test our system we need to find some suitable test cases
that are not biased towards our own system. We have chosen to first of
all test our system against a subset of the disease cases in BMJ
\cite{HangwiTang11102006}, i.e. disease cases which can be found in
our system\footnote{As mentioned earlier in \ref{Database} our system
  can only help diagnose the diseases contained in the
  system}. However, there is one major difference between the tests
conducted by the people behind the BMJ article - they have a medical
background (a respiratory and sleep physician and a rheumatologists)
contrary to our computer science background. This means that we have
no bias or knowledge about selecting symptoms and --- as they explain ---
given some of the symptoms the correct diagnosis were evident to
them. Note that we will --- in the following sections --- be referring to
the subset of the test cases as BMJ since this is were it was
found. \\

The subset of the BMJ test cases include the following 13 diseases (Table \ref{BMJCases}):

\begin{table}[!h]
\caption{Disease / Symptoms list}
\label{BMJCases}
\begin{tabular}{|l|p{7cm}|}
\hline
Disease & Symptoms \\
\hline
Infective endocarditis & Acute, aortic,  regurgitation, depression,  abscess \\
\hline
Cushing's syndrome & hypertension, adrenal, mass \\
\hline
Eosinophilic granuloma & Hip, lesion, older, child \\
\hline
Ehrlichiosis & fever, bilateral, thigh, pain, weakness \\
\hline
Neurofibromatosis type 1 & multiple, spinal, tumours, skin, tumours \\
\hline
Pheochromocytoma & hypertension, papilledema, headache, renal, mass, cafe, au, lait \\
\hline
Creutzfeldt-Jakob disease & ataxia, confusion, insomnia, death \\
\hline
Churg-Strauss syndrome & Wheeze, weight, loss, ANCA, haemoptysis, haematuria \\
\hline
Dermatomyositis & myopathy, neoplasia, dysphagia, rash, periorbital, swelling \\
\hline
Cat Scratch Disease & renal, transplant, fever, cat, lymphadenopathy \\
\hline
TEN & bullous, skin, conditions, respiratory, failure, carbamazepine \\
\hline
MELAS & seizure, confusion, dysphasia, T2, lesions \\
\hline
Brugada syndrome & cardiac arrest sleep \\
\hline
\end{tabular}
\end{table}

\subsubsection{Orpha.net}
To examine significance of the test result, we have additionally
selected some diseases at 'random' from Orpha.net. We require that the
disease has a description on Orpha.net containing a sentence with
'characterized by'. Occationally we have meant that the 'characterized
by' contained too many specific symptoms (e.g. derivatives of the name
of the disease or a several sentences long list of symptoms) and have
removed certain symptoms from the list. Examples of reductions would
be\footnote{Note that this is based solely on our own judgement as
  non-physicians.}\\

{\small
\textit{congenital anomalies (microcephaly, specific facial characteristics, broad thumbs and halluces and postnatal growth retardation), intellectual deficit and behavioural characteristics}
}\\

has been reduced to\\

{\small
\textit{congenital anomalies, intellectual deficit, behavioural}
}\\

and\\

{\small
\textit{congenital malformations: hydrocephalus (due to Dandy-Walker anomaly), cleft palate, and severe joint contractures}
}\\

has been reduced to \\

{\small
\textit{congenital malformations: hydrocephalus, cleft palate, severe joint contractures}
}\\

The test cases from Orpha.net include 30 different diseases can be found in table \ref{OrphanetCases1} and \ref{OrphanetCases2}.

\begin{table}[H]
\caption{Disease / symptom list 2}
\label{OrphanetCases1}
\begin{tabular}{| p{6.5cm} | p{6.5cm} |}
\hline
Disease name & Symptom list \\
\hline
Apparent mineralocorticoid excess & early-onset, severe hypertension, associated, low renin levels, hypoaldosteronism \\
\hline
Rubinstein-Taybi syndrome & congenital anomalies, intellectual deficit, behavioural characteristics \\
\hline
Aagenaes syndrome & chronic severe lymphoedema, severe neonatal cholestasis, lessens during early childhood and becomes episodic \\
\hline
Aase Smith syndrome & congenital malformations: hydrocephalus, cleft palate, severe joint contractures \\
\hline
Achondroplasia & short limbs, hyperlordosis, short hands, macrocephaly, high forehead and saddle nose \\
\hline
Acalvaria & missing scalp and flat bones over an area of the cranial vault \\
\hline
Acrodysostosis & abnormally short and malformed bones of the hands and feet (peripheral dysostosis), nasal hypoplasia and mental retardation \\
\hline
Acromegaly & progressive somatic disfigurement (face and extremities) and systemic manifestations \\
\hline
Biliary atresia & biliary obstruction of unknown origin, neonatal period \\
\hline
Bronchiolitis obliterans with obstructive pulmonary disease & inflammatory and fibrosing thickening of bronchiolar walls, airflow obstruction \\
\hline
Cholera & severe diarrhea and vomiting \\
\hline
Choroideremia & progressive degeneration of the choroid, retinal pigment epithelium (RPE), and neural retina \\
\hline
Coats disease & abnormal development of retinal vessels (telangiectasia) with a progressive deposition of intraretinal or subretinal exudates \\
\hline
Omphalocele cleft palate syndrome lethal & omphalocele and cleft palate \\
\hline
Darier disease & keratotic papules in seborrheic areas and specific nail anomalies \\
\hline
Ichthyosis hepatosplenomegaly cerebellar degeneration & ichthyosis, hepatosplenomegaly and late-onset cerebellar ataxia \\
\hline
Emery-Dreifuss muscular dystrophy & muscular weakness and atrophy, with early contractures of the tendons and cardiomyopathy \\
\hline
\end{tabular}
\end{table}

\begin{table}[H]
\caption{Disease / symptom list 2, continued}
\label{OrphanetCases2}
\begin{tabular}{| p{6.5cm} | p{6.5cm}|}
\hline
Costello syndrome & postnatal growth retardation, coarse facies, intellectual deficit, skin anomalies and cardiac abnormalities \\
\hline
Fibrodysplasia ossificans progressiva & congenital malformation of great toes, progressive, disabling heterotopic osteogenesis in predictable anatomical patterns \\
\hline
Acropectorovertebral dysplasia & fusion of the carpal and tarsal bones, with complex anomalies of the fingers and toes \\
\hline
Osteogenesis imperfecta & increased bone fragility and low bone mass \\
\hline
Primary biliary cirrhosis & injury of the intrahepatic bile ducts \\
\hline
Hennekam syndrome & lymphoedema, intestinal lymphangiectasia, intellectual deficit and facial dysmorphism \\
\hline
Hyperlysinemia & elevated levels of lysine in the cerebrospinal fluid and blood \\
\hline
Jackson-Weiss syndrome & tarsal and/or metatarsal coalitions and variable craniosynostosis, accompanied by facial anomalies, broad halluces and normal hands \\
\hline
Jalili syndrome & amelogenesis imperfecta and cone-rod retinal dystrophy \\
\hline
Jeune syndrome & narrow thorax and short limbs \\
\hline
Multiple myeloma & overproduction of abnormal plasma cells in the bone marrow and manifested by skeletal destruction, bone pain, and presence of abnormous immunoglobulins \\
\hline
Trichodental syndrome & fine, dry and short hair with dental anomalies \\
\hline
\end{tabular}
\end{table}

\subsubsection{Blind tests}
In addition to the BMJ and Oprhanet test cases, we have performed a
blind test on disease cases given by Henrik \cite{TheDude}.  >>Finish
this section when ordentlig-syg results arrive!<<

\subsection{Scoring schemes}

As mentioned in the previous chapters, we will employ two different
kinds of scoring measures - the cosine and sum measure. The original
idea --- behind using a sum measure --- was to test how much the cosine
measure would outperform this simpler measure but as we shall see in
section \ref{TestingCosineSimilarity} and section
\ref{TestingSumSimilarity}, the cosine measure is actually
outperformed itself by the sum measure. We will try to explain this
'oddity' in the given section and for now focus the way we use the two
different kinds of measure. The following cosine and sum score
measures are described in accordance to how they function on the term
document matrix. The exception of the disease matrix is described at
the end of this section.

\subsubsection{The cosine score\label{CosineScore}}

We will be testing the following three different approaches to using
the cosine similarity measure: cosine mean, cosine median and cosine
max.

\subsubsection{Cosine mean}
Every disease has one or more documents attached to it (as described
in section \ref{Database}). This means that the same disease might be
returned many times when looking at a top score of document similarity
measures produced by e.g. the cosine score. Therefore --- to give each
disease a score --- we use a form of concensus method where we sum the
scores of each document belonging to that disease. This produces a
mean score of each disease. \\

When the system (or more specifically the Query module) receives a
query, it ranks the query vector of terms against all document vectors
in which one or more of the terms has appeared. This results in a list
of scores $\mathbf{x}_{\textrm{all diseases}} = \left\{x_1, x_2,
\dots, x_n \right\}$. It then runs through every scored document and
adds the score to the disease from which the document came (in
accordance to the concensus method just described). Since some
documents appear in more than one disease (\ref{Database}), several
diseases might have the sum of a single document added to its score
$\mathbf{x}_{\textrm{disease}_{1}} = \left\{x_{\textrm{sum for} x_2,
  x_7, x_i, \dots, x_j}\right\}$, $\mathbf{x}_{\textrm{disease}_{2}} =
\left\{x_{\textrm{sum for} x_1, x_2, x_9, \dots, x_47, x_n}\right\},
\dots$.  Lastly, we evaluate the total ranking of each disease. We
combine each $x_{\textrm{disease}_1}$, $x_{\textrm{disease}_2}$ into a
list of all the returned disease $\mathbf{SL} =
\left\{x_{\textrm{disease}_1},x_{\textrm{disease}_2}, \dots,
x_{\textrm{disease}_n}\right\}$. These are then sorted and the highest
scoring is deemed the most likely to be the correct disease given the
query vector.

\subsubsection{Cosine median}
The median is calculated much like the mean, except for selecting the
median of $\mathbf{x}_{\textrm{disease}_{1}} = \left\{x_2,x_7, x_i,
\dots, x_j\right\}, \mathbf{x}_{\textrm{disease}_{2}}, \dots,
\mathbf{x}_{\textrm{disease}_{n}}$, instead of summing the scores as
we did above.

\subsubsection{Cosine max}
Does the same as above, except that it selects the maximum scoring in each
disease lists and sort the resulting list and select the highest
scoring as the most probable.

\subsubsection{The sum score}

The sum measure works exactly like the cosine mean measure, except for
running on non-normalized vectors. See \ref{VectorSimilarity} for
reasoning.

\subsubsection{The disease matrix exception}

This is simply a short note on how we use the cosine and sum measures
on the disease matrix. The disease matrix has no document vectors and
is solely made up of summed disease vectors. This means that there is
no point in using cosine mean, median or max, as there is no multiple
label occurences to run a concensus method over. Here the score is
simply the cosine or sum measure calculated for each of the diseases
that contain the queried term(s).

\subsection{Testing the cosine similarity measure\label{TestingCosineSimilarity}}

The first test we run is on for the three different cosine scoring measures - mean, 
median and max. On the two barcharts below \ref{termDoc_bmj_hist_3000_ns_mea_med_max_nc} 
and \ref{termDoc_orphan_hist_3000_ns_mea_med_max_nc} the query scores of the BMJ 
and the Orhpa.net test cases are shown. These are run on the non-stemmed term document matrix. 
The scores a drawn on a logarithmic scale while the 'real' scores a shown below each chart. 
Note that the values are 0-indexed(!) and all tests are performed on TF-IDF preprocessed matrices. \\

\begin{figure}[H]
        \begin{center}
          \includegraphics[width=1.0\textwidth]{barcharts/termDoc_bmj_hist_3000_ns_mea_med_max_nc.png}
        \end{center}
        \caption{Test for mean, median and max cosine measure on a non-stemmed term-doc}
        \label{termDoc_bmj_hist_3000_ns_mea_med_max_nc}
\end{figure}

\begin{table}[H]
  \begin{tiny}
  \label{testResult}
  \begin{tabular}{|l|r|r|r|r|r|r|r|}
    \hline
    Measure &Infec&Cushi&Eosin&Ehrli&Neuro&Pheoc&Creut \\
    \hline
    Cosine: mean & 25 & 2 & 66 & 692 & 128 & 139 & 128 \\
    \hline
    Cosine: median & 123 & 179 & 1210 & 1004 & 665 & 502 & 76 \\
    \hline
    Cosine: max & 28 & 7 & 311 & 1123 & 166 & 375 & 109  \\
    \hline
  \multicolumn{8}{c}{} \\
  \end{tabular}
  \begin{tabular}{|l|r|r|r|r|r|r|r|}
    \hline
    Measure &Churg&Derma&Cat S&TEN&MELAS&Bruga& \scriptsize{\textbf{\# in top 20}} \\
    \hline
    Cosine: mean non-stemmed & 3 & 63 & 0 & 2 & 52 & 9 & \scriptsize{\textbf{5}} \\
    \hline
    Cosine: mean stemmed & 343 & 455 & 59 & 392 & 430 & 464 &  \scriptsize{\textbf{0}}\\
    \hline
    Cosine: max non-stemmed & 16 & 21 & 0 & 39 & 96 & 47 & \scriptsize{\textbf{3}} \\
    \hline
  \end{tabular}
  \end{tiny}
\end{table}

\begin{figure}[H]
        \begin{center}
          \includegraphics[width=1.0\textwidth]{barcharts/termDoc_orphan_hist_3000_ns_mea_med_max_nc.png}
        \end{center}
        \caption{Test of mean, median and max cosine measure on a non-stemmed term-doc}
        \label{termDoc_orphan_hist_3000_ns_mea_med_max_nc}
\end{figure}

\begin{table}[H]
\begin{tiny}
  \begin{tabular}{|l|r|r|r|r|r|r|r|r|r|r|r|r|r|r|r|r|r|r|r|r|r|r|r|r|r|r|r|r|r|r|r|}
    \hline
    Measure &Ap&Ru&Aa&Aa&Ac&Ac&Ac&Ac&Bi&Br&Ch&Ch&Co&Om&Da\\
    \hline
    Cosine: mean & 4 & 664 & 30 & 47 & 38 & 85 & 62 & 1371 & 1 & 32 & 83 & 15 & 0 & 26 & 2 \\
    \hline
    Cosine: median & 163 & 357 & 76 & 240 & 948 & 4 & 76 & 141 & 384 & 314 & 505 & 211 & 44 & 181 & 42 \\
    \hline
    Cosine: max & 4 & 858 & 0 & 10 & 44 & 15 & 2 & 541 & 4 & 116 & 99 & 18 & 0 & 6 & 2\\
    \hline
    \multicolumn{16}{c}{} \\
    \end{tabular}
    \begin{tabular}{|l|r|r|r|r|r|r|r|r|r|r|r|r|r|r|r|r|r|r|r|r|r|r|r|r|r|r|r|r|r|r|}
    \hline
     Measure &Ic&Em&Co&Fi&Ac&Os&Pr&He&Hy&Ja&Ja&Je&Ja&Mu&Tr &  \scriptsize{\textbf{\# in top 20}} \\
    \hline
     Cosine: mean & 81 & 3 & 16 & 2 & 3000 & 4 & 10 & 13 & 24 & 66 & 35 & 3 & 66 & 4 & 34 & \scriptsize{\textbf{13}} \\
    \hline
    Cosine: median & 773 & 87 & 169 & 189 & 3000 & 179 & 265 & 21 & 491 & 692 & 37 & 435 & 692 & 358 & 233 & \scriptsize{\textbf{1}} \\
    \hline
     Cosine: max  & 0 & 22 & 0 & 3 & 3000 & 9 & 67 & 5 & 63 & 201 & 1 & 8 & 201 & 9 & 0 & \scriptsize{\textbf{19}} \\
     \hline
  \end{tabular}
\end{tiny}
\end{table}

As we see here, the mean cosine measure performs best in the BMJ test
set while the max cosine measure scores best in the Orpha.net test
set. The median measure has an overall low score and running some
quick tests on the different matrices, quickly reveals that median is
not well suited as a measure to take into consideration. Therefore we
will not be testing further on the cosine median score and continues
with the two remaining scores from here on. Note that the score that
has the worst performance in the Orpha.net test. It can and will
happen that diseases are not found within the top 3000 documents that
is returned. When this is the case --- to avoid confusion in the bar
charts and statistics --- we simply set the score of any disease not
found to the value 3000, representing a bad performance. Note also
that a missing bar represents the top score 0. \\

We now continue testing the scoring measures, this time comparing the
non-stemmed and stemmed term document matrices. The results are shown
in the figures \ref{termDoc_bmj_hist_3000_ns_mea_s_mea_ns_max_s_max}
and \ref{termDoc_orphan_hist_3000_ns_mea_s_mea_ns_max_s_max} below.

\begin{figure}[H]
        \begin{center}
          \includegraphics[width=0.9\textwidth]{barcharts/termDoc_bmj_hist_3000_ns_mea_s_mea_ns_max_s_max.png}
        \end{center}
        \caption{Comparison of mean and max cosine measure tests on non-stemmed and stemmed term-doc matrices}
        \label{termDoc_bmj_hist_3000_ns_mea_s_mea_ns_max_s_max}
\end{figure}

\begin{table}[H]
  \begin{tiny}
  \label{testResult}
  \begin{tabular}{|l|r|r|r|r|r|r|r|}
    \hline
    Measure &Infec&Cushi&Eosin&Ehrli&Neuro&Pheoc&Creut \\
    \hline
    Cosine: mean non-stemmed &25&2&66&692&128&139&128 \\
    \hline
    Cosine: mean stemmed &24&2&110&710&292&113&110 \\
    \hline
    Cosine: max non-stemmed &28&7&311&1123&166&375&109 \\
    \hline
    Cosine: max stemmed &20&11&427&1232&210&370&108 \\
    \hline
  \multicolumn{8}{c}{} \\
  \end{tabular}
  \begin{tabular}{|l|r|r|r|r|r|r|r|}
    \hline
    Measure &Churg&Derma&Cat S&TEN&MELAS&Bruga& \scriptsize{\textbf{\# in top 20}} \\
    \hline
    Cosine: mean non-stemmed &3&63&0&2&52&9& \scriptsize{\textbf{5}} \\
    \hline
    Cosine: mean stemmed &3&38&0&2&51&9& \scriptsize{\textbf{5}}\\
    \hline
    Cosine: max non-stemmed &16&21&0&39&96&47& \scriptsize{\textbf{3}} \\
    \hline
    Cosine: max stemmed &28&22&0&62&192&56& \scriptsize{\textbf{2}} \\
    \hline
  \end{tabular}
  \end{tiny}
\end{table}

\begin{figure}[H]
        \begin{center}
          \includegraphics[width=0.9\textwidth]{barcharts/termDoc_orphan_hist_3000_ns_mea_s_mea_ns_max_s_max.png}
        \end{center}
        \caption{Comparison of mean and max cosine measure tests on non-stemmed and stemmed term-doc matrices}
        \label{termDoc_orphan_hist_3000_ns_mea_s_mea_ns_max_s_max}
\end{figure}

\begin{table}[H]
\begin{tiny}
  \begin{tabular}{|l|r|r|r|r|r|r|r|r|r|r|r|r|r|r|r|r|r|r|r|r|r|r|r|r|r|r|r|r|r|r|r|}
    \hline
    Measure &Ap&Ru&Aa&Aa&Ac&Ac&Ac&Ac&Bi&Br&Ch&Ch&Co&Om&Da\\
    \hline
    Cosine: mean non-stemmed &4&664&30&47&38&85&62&1371&1&32&83&15&0&26&2\\
    \hline
    Cosine: mean stemmed &4&248&29&48&23&106&64&1436&1&34&85&16&0&26&2\\
    \hline
    Cosine: max non-stemmed &4&858&0&10&44&15&2&541&4&116&99&18&0&6&2\\
    \hline
    Cosine: max stemmed &1&677&0&3&40&8&2&462&4&75&87&31&0&8&1 \\
    \hline
    \multicolumn{16}{c}{} \\
    \end{tabular}
    \begin{tabular}{|l|r|r|r|r|r|r|r|r|r|r|r|r|r|r|r|r|r|r|r|r|r|r|r|r|r|r|r|r|r|r|}
    \hline
     Measure &Ic&Em&Co&Fi&Ac&Os&Pr&He&Hy&Ja&Ja&Je&Ja&Mu&Tr &\scriptsize{\textbf{\# in top 20}} \\
    \hline
    Cosine: mean non-stemmed  &81&3&16&2&3000&4&10&13&24&66&35&3&66&4&34 & \scriptsize{\textbf{13}} \\
    \hline
    Cosine: mean stemmed &81&3&15&3&3000&4&11&11&24&60&46&3&60&7&36 & \scriptsize{\textbf{13}} \\
    \hline
    Cosine: max non-stemmed &81&3&15&3&3000&4&11&11&24&60&46&3&60&7&36 & \scriptsize{\textbf{19}} \\
    \hline
    Cosine: max stemmed &0&22&0&3&3000&9&67&5&63&201&1&8&201&9&0 & \scriptsize{\textbf{18}} \\
    \hline
  \end{tabular}
\end{tiny}
\end{table}

The two score tests just performed now presents us with a dilemma. In the BMJ test set 
the 'mean stemmed' and 'non-stemmed' scores performs best while in the Orpha.net test set, 
it is just the opposite. We have chosen to cope with this by taking out the top score 
measure for each of the test sets - 'mean non-stemmed' from BMJ and 'max stemmed' from 
the Orpha.net. \\

The next step is to analyse our data a bit by performing a square root transformation 
\ref{SquareRoot} of the TF-IDF preprocessed data above. Note that it is required that 
all values transformed are between 0 and 1 which in our case is secured by the fact 
that the matrices, we use for the cosine measure, are normalized. The reason for the 
square root analysis is that it allows us to see whether the data has been correctly 
weighted. The square root transformation raises small values by a greater degree than 
it does large values. This means that if our scores improve, the information containing 
terms in the term document matrix have not been given high enough values by the applied 
heuristics. \\

The tests are shown in the figures below, where we compare the best measures from above 
with their square root transformation.

\begin{figure}[H]
        \begin{center}
          \includegraphics[width=0.9\textwidth]{barcharts/termDoc_bmj_hist_3000_ns_mea_ns_mea_sqr_s_max_s_max_sqr.png}
        \end{center}
        \caption{Comparison of mean (non-stemmed term-doc) and max (stemmed term-doc) cosine measure tests with and without sqrt-transformation}
        \label{termDoc_bmj_hist_3000_ns_mea_ns_mea_sqr_s_max_s_max_sqr}
\end{figure}

\begin{table}[H]
  \begin{tiny}
  \label{testResult}
  \begin{tabular}{|l|r|r|r|r|r|r|r|}
    \hline
    Measure &Infec&Cushi&Eosin&Ehrli&Neuro&Pheoc&Creut \\
    \hline
    Cosine: mean non-stemmed no sqrt &25&2&66&692&128&139&128 \\
    \hline
    Cosine: mean non-stemmed sqrt &19&3&22&1268&115&105&108 \\
    \hline
    Cosine: max stemmed no sqrt &20&11&427&1232&210&370&108 \\
    \hline
    Cosine: max stemmed sqrt &2&10&136&1123&68&249&130 \\
    \hline
  \multicolumn{8}{c}{} \\
  \end{tabular}
  \begin{tabular}{|l|r|r|r|r|r|r|r|}
    \hline
    Measure &Churg&Derma&Cat S&TEN&MELAS&Bruga& \scriptsize{\textbf{\# in top 20}} \\
    \hline
    Cosine: mean non-stemmed no sqrt &3&63&0&2&52&9 &\scriptsize{\textbf{5}} \\
    \hline
    Cosine: mean non-stemmed sqrt &5&54&0&2&41&7 &\scriptsize{\textbf{6}}\\
    \hline
    Cosine: max stemmed no sqrt &28&22&0&62&192&56 & \scriptsize{\textbf{2}} \\
    \hline
    Cosine: max stemmed sqrt &44&8&0&47&65&25 & \scriptsize{\textbf{4}} \\
    \hline
  \end{tabular}
  \end{tiny}
\end{table}

\begin{figure}[H]
        \begin{center}
          \includegraphics[width=0.9\textwidth]{barcharts/termDoc_orphan_hist_3000_ns_mea_ns_mea_sqr_s_max_s_max_sqr.png}
        \end{center}
        \caption{Comparison of mean (non-stemmed term-doc) and max (stemmed term-doc) cosine measure tests with and without sqrt-transformation}
        \label{termDoc_orphan_hist_3000_ns_mea_ns_mea_sqr_s_max_s_max_sqr}
\end{figure}

\begin{table}[H]
\begin{tiny}
  \begin{tabular}{|l|r|r|r|r|r|r|r|r|r|r|r|r|r|r|r|r|r|r|r|r|r|r|r|r|r|r|r|r|r|r|r|}
    \hline
    Measure &Ap&Ru&Aa&Aa&Ac&Ac&Ac&Ac&Bi&Br&Ch&Ch&Co&Om&Da\\
    \hline
    Cosine: mean non-stemmed no-sqrt &4&664&30&47&38&85&62&1371&1&32&83&15&0&26&2\\
    \hline
    Cosine: mean non-stemmed sqrt &4&725&75&37&38&85&68&1651&1&23&80&15&0&26&2\\
    \hline
    Cosine: max stemmed no-sqrt &1&677&0&3&40&8&2&462&4&75&87&31&0&8&1\\
    \hline
    Cosine: max stemmed sqrt &12&145&0&0&93&6&2&1842&6&25&44&15&0&15&1 \\
    \hline
    \multicolumn{16}{c}{} \\
    \end{tabular}
    \begin{tabular}{|l|r|r|r|r|r|r|r|r|r|r|r|r|r|r|r|r|r|r|r|r|r|r|r|r|r|r|r|r|r|r|}
    \hline
     Measure &Ic&Em&Co&Fi&Ac&Os&Pr&He&Hy&Ja&Ja&Je&Ja&Mu&Tr &\scriptsize{\textbf{\# in top 20}} \\
    \hline
    Cosine: mean non-stemmed no-sqrt &81&3&16&2&3000&4&10&13&24&66&35&3&66&4&34 & \scriptsize{\textbf{13}} \\
    \hline
    Cosine: mean non-stemmed sqrt &218&3&13&2&3000&1&9&14&78&84&48&3&84&1&62 & \scriptsize{\textbf{13}} \\
    \hline
    Cosine: max stemmed no sqrt &0&17&0&3&3000&10&58&5&86&162&1&24&162&9&0 & \scriptsize{\textbf{18}} \\
    \hline
    Cosine: max stemmed sqrt &0&10&0&3&3000&7&46&4&128&115&1&24&115&2&1 & \scriptsize{\textbf{19}} \\
    \hline
  \end{tabular}
\end{tiny}
\end{table}

These tests reveal some interesting results. Looking at the BMJ test set we see an overall 
improvement in the performance of the square root transformed measures. In Orpha.net test 
set there is an improvement in 'max stemmed' measure while a slight worsening of the 'mean 
non-stemmed' measure. However, there is no change in the number of top 20 results and the 
other measures shows a more significant improvement that the worsening of the last mentioned 
measure. Based on these results, we will not deny that the data in the TF-IDF matrices are 
not as optimized as could have been expected. But we can not say if these anomalies stem from 
the data or the calculations themselves. For now, we choose to view the square root 
transformation as a general improvement. \\

In section \ref{DiseaseMatrix}, we will be using the best measure of the cosine scoring tests 
executed above - the 'mean stemmed sqrt' and the 'max stemmed sqrt' cosine similarity measures.

\subsection{Testing the sum similarity measure\label{TestingSumSimilarity}}

In this section, we perform the same tests as described in the previous section, except for the square 
root transformation which makes no sense since we will be running on unnormalized data. Or in 
other word on values above and below 1 \ref{SquareRoot}. The first test is run for the mean, 
median and max sum measures on a TF-IDF non-normalized term document matrix. The results are 
shown on the figures \ref{termDoc_bmj_hist_3000_sum_mea_med_max} below. 

\begin{figure}[H]
        \begin{center}
          \includegraphics[width=0.9\textwidth]{barcharts/termDoc_bmj_hist_3000_sum_mea_med_max.png}
        \end{center}
        \caption{Test of mean, median and max sum measure on a non-stemmed term-doc matrix}
        \label{termDoc_bmj_hist_3000_sum_mea_med_max}
\end{figure}

\begin{table}[H]
  \begin{tiny}
  \label{testResult}
  \begin{tabular}{|l|r|r|r|r|r|r|r|}
    \hline
    Measure &Infec&Cushi&Eosin&Ehrli&Neuro&Pheoc&Creut \\
    \hline
    Sum: mean &23&3&362&772&35&76&144 \\
    \hline
    Sum: median &188&459&2150&1878&213&852&974 \\
    \hline
    Sum: max &54&10&344&2401&235&469&495  \\
    \hline
  \multicolumn{8}{c}{} \\
  \end{tabular}
  \begin{tabular}{|l|r|r|r|r|r|r|r|}
    \hline
    Measure &Churg&Derma&Cat S&TEN&MELAS&Bruga& \scriptsize{\textbf{\# in top 20}} \\
    \hline
    Sum: mean non-stemmed &1&45&0&25&53&5& \scriptsize{\textbf{4}} \\
    \hline
    Sum: mean stemmed &83&1353&670&2193&689&1210 &  \scriptsize{\textbf{0}}\\
    \hline
    Sum: max non-stemmed &70&6&19&441&391&26 & \scriptsize{\textbf{3}} \\
    \hline
  \end{tabular}
  \end{tiny}
\end{table}

\begin{figure}[H]
        \begin{center}
          \includegraphics[width=0.9\textwidth]{barcharts/termDoc_orphan_hist_3000_ns_mea_med_max_sum.png}
        \end{center}
        \caption{Test of mean, median and max sum measure on a non-stemmed term-doc matrix}
        \label{termDoc_orphan_hist_3000_ns_mea_med_max_sum}
\end{figure}

\begin{table}[H]
\begin{tiny}
  \begin{tabular}{|l|r|r|r|r|r|r|r|r|r|r|r|r|r|r|r|r|r|r|r|r|r|r|r|r|r|r|r|r|r|r|r|}
    \hline
    Measure &Ap&Ru&Aa&Aa&Ac&Ac&Ac&Ac&Bi&Br&Ch&Ch&Co&Om&Da\\
    \hline
    Sum: mean &6&910&917&32&122&460&145&3119&1&21&342&50&0&45&2 \\
    \hline
    Sum: median &626&2814&495&219&1232&963&182&3590&872&1207&1056&595&526&940&179 \\
    \hline
    Sum: max &119&2081&611&113&1031&48&203&3833&7&127&1139&109&9&7&2\\
    \hline
    \multicolumn{16}{c}{} \\
    \end{tabular}
    \begin{tabular}{|l|r|r|r|r|r|r|r|r|r|r|r|r|r|r|r|r|r|r|r|r|r|r|r|r|r|r|r|r|r|r|}
    \hline
     Measure &Ic&Em&Co&Fi&Ac&Os&Pr&He&Hy&Ja&Ja&Je&Ja&Mu&Tr &  \scriptsize{\textbf{\# in top 20}} \\
    \hline
     Sum: mean &137&3&44&14&2458&0&9&36&37&132&47&26&132&37&127 & \scriptsize{\textbf{8}} \\
    \hline
    Sum: median &1292&503&408&304&845&320&204&143&1165&1763&19&467&1763&1532&100& \scriptsize{\textbf{1}} \\
    \hline
     Sum: max  &357&7&401&3&1957&13&0&102&169&260&4&198&260&72&55 & \scriptsize{\textbf{9}} \\
     \hline
  \end{tabular}
\end{tiny}
\end{table}

Like in the previous section, we again see the poor results given by the median measure and discards 
this for further testing. In the next tests, we compare the mean and sum measure in the stemmed and 
non-stemmed matrices. The tests are shown on the figures \ref{termDoc_bmj_hist_3000_ns_s_mea_max_sum} 
and \ref{termDoc_orphan_hist_3000_ns_s_mea_max_sum}.

\begin{figure}[H]
        \begin{center}
          \includegraphics[width=0.9\textwidth]{barcharts/termDoc_bmj_hist_3000_ns_s_mea_max_sum.png}
        \end{center}
        \caption{Comparison of mean and max sum measure tests on non-stemmed and stemmed term-doc matrices}
        \label{termDoc_bmj_hist_3000_ns_s_mea_max_sum}
\end{figure} 

\begin{table}[H]
  \begin{tiny}
  \label{testResult}
  \begin{tabular}{|l|r|r|r|r|r|r|r|}
    \hline
    Measure &Infec&Cushi&Eosin&Ehrli&Neuro&Pheoc&Creut \\
    \hline
    Sum: mean non-stemmed &23&3&362&772&35&76&144 \\
    \hline
    Sum: mean stemmed &23&3&720&746&44&60&158 \\
    \hline
    Sum: max non-stemmed &54&10&344&2401&235&469&495 \\
    \hline
    Sum: max stemmed &120&6&10&2374&228&360&566 \\
    \hline
  \multicolumn{8}{c}{} \\
  \end{tabular}
  \begin{tabular}{|l|r|r|r|r|r|r|r|}
    \hline
    Measure &Churg&Derma&Cat S&TEN&MELAS&Bruga& \scriptsize{\textbf{\# in top 20}} \\
    \hline
    Sum: mean non-stemmed &1&45&0&25&53&5 &\scriptsize{\textbf{4}} \\
    \hline
    Sum: mean stemmed &1&33&0&27&88&5 &\scriptsize{\textbf{4}}\\
    \hline
    Sum: max non-stemmed &70&6&19&441&391&26 & \scriptsize{\textbf{3}} \\
    \hline
    Sum: max stemmed &62&8&7&394&496&26 & \scriptsize{\textbf{4}} \\
    \hline
  \end{tabular}
  \end{tiny}
\end{table}

\begin{figure}[H]
        \begin{center}
          \includegraphics[width=0.9\textwidth]{barcharts/termDoc_orphan_hist_3000_ns_s_mea_max_sum.png}
        \end{center}
        \caption{Comparison of mean and max sum measure tests on non-stemmed and stemmed term-doc matrices}
        \label{termDoc_orphan_hist_3000_ns_s_mea_max_sum}
\end{figure} 

\begin{table}[H]
\begin{tiny}
  \begin{tabular}{|l|r|r|r|r|r|r|r|r|r|r|r|r|r|r|r|r|r|r|r|r|r|r|r|r|r|r|r|r|r|r|r|}
    \hline
    Measure &Ap&Ru&Aa&Aa&Ac&Ac&Ac&Ac&Bi&Br&Ch&Ch&Co&Om&Da\\
    \hline
    Sum: mean non-stemmed &6&910&917&32&122&460&145&3119&1&21&342&50&0&45&2\\
    \hline
    Sum: mean stemmed &6&708&644&28&97&460&170&2522&1&30&190&58&0&43&6\\
    \hline
    Sum: max non-stemmed &119&2081&611&113&1031&48&203&3833&7&127&1139&109&9&7&2\\
    \hline
    Sum: max stemmed &75&2228&638&113&993&48&171&3281&3&131&1194&131&7&7&3 \\
    \hline
    \multicolumn{16}{c}{} \\
    \end{tabular}
    \begin{tabular}{|l|r|r|r|r|r|r|r|r|r|r|r|r|r|r|r|r|r|r|r|r|r|r|r|r|r|r|r|r|r|r|}
    \hline
     Measure &Ic&Em&Co&Fi&Ac&Os&Pr&He&Hy&Ja&Ja&Je&Ja&Mu&Tr &\scriptsize{\textbf{\# in top 20}} \\
    \hline
    Sum: mean non-stemmed &137&3&44&14&2458&0&9&36&37&132&47&26&132&37&127 & \scriptsize{\textbf{8}} \\
    \hline
    Sum: mean stemmed &136&3&39&16&2636&0&8&50&33&115&53&31&115&121&124 & \scriptsize{\textbf{8}} \\
    \hline
    Sum: max non-stemmed &357&7&401&3&1957&13&0&102&169&260&4&198&260&72&55 & \scriptsize{\textbf{9}} \\
    \hline
    Sum: max stemmed &365&7&414&3&2242&253&0&150&188&42&4&6&42&372&55 & \scriptsize{\textbf{9}} \\
    \hline
  \end{tabular}
\end{tiny}
\end{table}

We see here that the 'mean sum' similarity measure clearly outperforms the 'max sum'. 
In section \ref{DiseaseMatrix}, we compare this measure with the best of the cosine 
measure on and the term document and disease matrices.

\subsection{Disease and term document matrix - cosine, and sum and final result}

Now that we have found the results for the best measures to be used on
the term document matrix, we focus our attention on the disease
matrix. We will in the following be testing the sum and cosine measure
on the disease matrix and --- in the end of the section --- compare these
results to that of the term document. \\

In the first test, we look at the performance of the cosine (mean),
cosine-sqrt and the sum measure on the BMJ and Orpha-net test
sets. These test are performed on both the non-stemmed and
stemmed. The results are shown on the figures
\ref{diseaseMatrix_bmj_hist_norm_3000_ns_cos_sqrt_cos_sum_nn},
\ref{diseaseMatrix_orphan_hist_NOTnorm_3000_ns_cos_sqrt_cos_sum_nn},
\ref{diseaseMatrix_bmj_hist_norm_3000_s_cos_sqrt_cos_sum_nn} and
\ref{diseaseMatrix_orphan_hist_NOTnorm_3000_s_cos_sqrt_cos_sum_nn}
below: \\

\textbf{Non-stemmed:}

\begin{figure}[H]
        \begin{center}
          \includegraphics[width=0.9\textwidth]{barcharts/diseaseMatrix_bmj_hist_norm_3000_ns_cos_sqrt_cos_sum_nn.png}
        \end{center}
        \caption{Test of cosine measure (with and without sqrt-transformation) and sum measure on a non-stemmed disease matrix}
        \label{diseaseMatrix_bmj_hist_norm_3000_ns_cos_sqrt_cos_sum_nn}
\end{figure}

\begin{table}[H]
  \begin{tiny}
  \label{testResult}
  \begin{tabular}{|l|r|r|r|r|r|r|r|}
    \hline
    Measure &Infec&Cushi&Eosin&Ehrli&Neuro&Pheoc&Creut \\
    \hline
    Cosine no sqrt non-stemmed &31&62&474&2220&377&1225&93 \\
    \hline
    Cosine sqrt non-stemmed &16&37&375&2001&270&1037&25 \\
    \hline
    Sum non-stemmed &6&1&323&691&30&427&8  \\
    \hline
  \multicolumn{8}{c}{} \\
  \end{tabular}
  \begin{tabular}{|l|r|r|r|r|r|r|r|}
    \hline
    Measure &Churg&Derma&Cat S&TEN&MELAS&Bruga& \scriptsize{\textbf{\# in top 20}} \\
    \hline
    Cosine no sqrt non-stemmed &8&20&5&227&118&94 &\scriptsize{\textbf{2}} \\
    \hline
    Cosine sqrt non-stemmed &6&0&1&97&45&25 &  \scriptsize{\textbf{4}}\\
    \hline
    Sum non-stemmed &2&4&0&3&19&6 & \scriptsize{\textbf{9}} \\
    \hline
  \end{tabular}
  \end{tiny}
\end{table}

\begin{figure}[H]
        \begin{center}
          \includegraphics[width=0.9\textwidth]{barcharts/diseaseMatrix_orphan_hist_NOTnorm_3000_ns_cos_sqrt_cos_sum_nn.png}
        \end{center}
        \caption{Test of cosine measure (with and without sqrt-transformation) and sum measure on a non-stemmed disease matrix}
        \label{diseaseMatrix_orphan_hist_NOTnorm_3000_ns_cos_sqrt_cos_sum_nn}
\end{figure}

\begin{table}[H]
\begin{tiny}
  \begin{tabular}{|l|r|r|r|r|r|r|r|r|r|r|r|r|r|r|r|r|r|r|r|r|r|r|r|r|r|r|r|r|r|r|r|}
    \hline
    Measure &Ap&Ru&Aa&Aa&Ac&Ac&Ac&Ac&Bi&Br&Ch&Ch&Co&Om&Da\\
    \hline
    Cosine no sqrt non-stemmed &95&599&0&76&307&99&47&2989&28&3&430&165&7&37&11 \\
    \hline
    Cosine sqrt non-stemmed &76&257&0&20&122&41&9&1912&16&1&448&128&1&13&5 \\
    \hline
    Sum non-stemmed &9&123&4&10&4&81&109&601&3&7&68&9&0&2&1\\
    \hline
    \multicolumn{16}{c}{} \\
    \end{tabular}
    \begin{tabular}{|l|r|r|r|r|r|r|r|r|r|r|r|r|r|r|r|r|r|r|r|r|r|r|r|r|r|r|r|r|r|r|}
    \hline
     Measure &Ic&Em&Co&Fi&Ac&Os&Pr&He&Hy&Ja&Ja&Je&Ja&Mu&Tr &  \scriptsize{\textbf{\# in top 20}} \\
    \hline
     Cosine no sqrt non-stemmed &7&97&717&150&562&74&64&31&5&89&1&75&89&222&1 & \scriptsize{\textbf{8}} \\
    \hline
     Cosine sqrt non-stemmed &0&33&380&20&1687&39&47&26&4&33&1&83&33&46&0 &\scriptsize{\textbf{11}} \\
    \hline
     Sum non-stemmed &6&0&3&0&3000&0&8&5&107&46&4&2&46&2&55 & \scriptsize{\textbf{20}} \\
     \hline
  \end{tabular}
\end{tiny}
\end{table}

\textbf{Stemmed:}

\begin{figure}[H]
        \begin{center}
          \includegraphics[width=0.9\textwidth]{barcharts/diseaseMatrix_bmj_hist_norm_3000_s_cos_sqrt_cos_sum_nn.png}
        \end{center}
        \caption{Test of cosine measure (with and without sqrt-transformation) and sum measure on a stemmed disease matrix}
        \label{diseaseMatrix_bmj_hist_norm_3000_s_cos_sqrt_cos_sum_nn}
\end{figure}

\begin{table}[H]
  \begin{tiny}
  \label{testResult}
  \begin{tabular}{|l|r|r|r|r|r|r|r|}
    \hline
    Measure &Infec&Cushi&Eosin&Ehrli&Neuro&Pheoc&Creut \\
    \hline
    Cosine no sqrt stemmed &37&63&872&1963&533&1198&93 \\
    \hline
    Cosine sqrt stemmed &25&37&748&1970&384&1053&25 \\
    \hline
    Sum stemmed &5&1&597&511&29&413&6 \\
    \hline
  \multicolumn{8}{c}{} \\
  \end{tabular}
  \begin{tabular}{|l|r|r|r|r|r|r|r|}
    \hline
    Measure &Churg&Derma&Cat S&TEN&MELAS&Bruga& \scriptsize{\textbf{\# in top 20}} \\
    \hline
    Cosine no sqrt stemmed &4&18&9&230&221&91 &\scriptsize{\textbf{3}} \\
    \hline
    Cosine sqrt stemmed &2&0&1&102&68&30 &  \scriptsize{\textbf{3}}\\
    \hline
    Sum stemmed &1&3&0&3&26&6 & \scriptsize{\textbf{8}} \\
    \hline
  \end{tabular}
  \end{tiny}
\end{table}

\begin{figure}[H]
        \begin{center}
          \includegraphics[width=0.9\textwidth]{barcharts/diseaseMatrix_orphan_hist_NOTnorm_3000_s_cos_sqrt_cos_sum_nn.png}
        \end{center}
        \caption{Test of cosine measure (with and without sqrt-transformation) and sum measure on a stemmed disease matrix}
        \label{diseaseMatrix_orphan_hist_NOTnorm_3000_s_cos_sqrt_cos_sum_nn}
\end{figure}

\begin{table}[H]
\begin{tiny}
  \begin{tabular}{|l|r|r|r|r|r|r|r|r|r|r|r|r|r|r|r|r|r|r|r|r|r|r|r|r|r|r|r|r|r|r|r|}
    \hline
    Measure &Ap&Ru&Aa&Aa&Ac&Ac&Ac&Ac&Bi&Br&Ch&Ch&Co&Om&Da\\
    \hline
    Cosine no sqrt stemmed &94&553&0&80&284&94&51&1454&32&1&433&181&10&33&18 \\
    \hline
    Cosine sqrt stemmed &74&263&0&19&106&44&12&635&19&1&446&133&4&13&4 \\
    \hline
    Sum stemmed &9&90&17&9&4&86&79&105&2&7&64&16&0&2&1\\
    \hline
    \multicolumn{16}{c}{} \\
    \end{tabular}
    \begin{tabular}{|l|r|r|r|r|r|r|r|r|r|r|r|r|r|r|r|r|r|r|r|r|r|r|r|r|r|r|r|r|r|r|}
    \hline
     Measure &Ic&Em&Co&Fi&Ac&Os&Pr&He&Hy&Ja&Ja&Je&Ja&Mu&Tr &  \scriptsize{\textbf{\# in top 20}} \\
    \hline
     Cosine no sqrt stemmed &7&169&710&334&704&167&65&26&8&74&1&83&74&468&1 & \scriptsize{\textbf{8}} \\
    \hline
     Cosine sqrt stemmed &0&80&391&122&2137&74&54&12&2&30&1&99&30&162&0 &\scriptsize{\textbf{13}} \\
    \hline
     Sum stemmed &6&0&2&0&3000&1&9&3&84&39&3&1&39&2&59 & \scriptsize{\textbf{20}} \\
     \hline
  \end{tabular}
\end{tiny}
\end{table}

When it comes to scoring diseases on in the disease matrix, the sum measure greatly outrival the 
cosine measure - with or without the square root transformation. If we look at the average values 
of the returned results, it seems that the stemmed version of the disease matrix is the best choice 
for optimized performance. \\

For the final test of measure and model, we compare the top results of the two matrices - term 
document and disease matrix. We will compare the different scores from the stemmed version of 
both matrix types since this seems to provide the overall best performance. In the figures 
\ref{termDoc_bmj_hist_3000_sum_dm_mea_cos_sqrt_td_max_cos_sqrt_td_mea_sum_td} and 
\ref{termDoc_orphan_hist_3000_sum_dm_mea_cos_sqrt_td_max_cos_sqrt_td_mea_sum_nn_td} below are 
bar chart of the best scores found for the prototype system:

\begin{figure}[H]
        \begin{center}
          \includegraphics[width=0.9\textwidth]{barcharts/termDoc_bmj_hist_3000_sum_dm_mea_cos_sqrt_td_max_cos_sqrt_td_mea_sum_td.png}
        \end{center}
        \caption{Comparison of the mean sum measure, and mean and max sqrt cosine measure, on a stemmed term-doc matrix, and of the sum measure on a stemmed disease matrix}
        \label{termDoc_bmj_hist_3000_sum_dm_mea_cos_sqrt_td_max_cos_sqrt_td_mea_sum_td}
\end{figure}

\begin{table}[H]
  \begin{tiny}
  \label{testResult}
  \begin{tabular}{|l|r|r|r|r|r|r|r|}
    \hline
    Measure &Infec&Cushi&Eosin&Ehrli&Neuro&Pheoc&Creut \\
    \hline
    Sum: disease matrix &5&1&597&511&29&413&6 \\
    \hline
    Cosine: term-doc mean-sqrt &19&3&22&1268&115&105&108 \\
    \hline
    Cosine: term-doc max-sqrt &2&10&136&1123&68&249&130 \\
    \hline
    Sum: term-doc mean &23&3&720&746&44&60&158 \\
    \hline
  \multicolumn{8}{c}{} \\
  \end{tabular}
  \begin{tabular}{|l|r|r|r|r|r|r|r|}
    \hline
    Measure &Churg&Derma&Cat S&TEN&MELAS&Bruga& \scriptsize{\textbf{\# in top 20}} \\
    \hline
    Sum: disease matrix &1&3&0&3&26&6 &\scriptsize{\textbf{8}} \\
    \hline
    Cosine: term-doc mean-sqrt &5&54&0&2&41&7 &\scriptsize{\textbf{6}}\\
    \hline
    Cosine: term-doc max-qrt &44&8&0&47&65&25 & \scriptsize{\textbf{4}} \\
    \hline
    Sum: term-doc mean &1&33&0&27&88&5 & \scriptsize{\textbf{4}} \\
    \hline
  \end{tabular}
  \end{tiny}
\end{table}

\begin{figure}[H]
        \begin{center}
          \includegraphics[width=0.9\textwidth]{barcharts/termDoc_orphan_hist_3000_sum_dm_mea_cos_sqrt_td_max_cos_sqrt_td_mea_sum_nn_td.png}
        \end{center}
        \caption{Comparison of the mean sum measure, and mean and max sqrt cosine measure, on a stemmed term-doc matrix, and of the sum measure on a stemmed disease matrix}
        \label{termDoc_orphan_hist_3000_sum_dm_mea_cos_sqrt_td_max_cos_sqrt_td_mea_sum_nn_td}
\end{figure}

\begin{table}[H]
\begin{tiny}
  \begin{tabular}{|l|r|r|r|r|r|r|r|r|r|r|r|r|r|r|r|r|r|r|r|r|r|r|r|r|r|r|r|r|r|r|r|}
    \hline
    Measure &Ap&Ru&Aa&Aa&Ac&Ac&Ac&Ac&Bi&Br&Ch&Ch&Co&Om&Da\\
    \hline
    Sum: disease matrix &9&90&17&9&4&86&79&105&2&7&64&16&0&2&1\\
    \hline
    Cosine: term-doc mean-sqrt &4&725&75&37&38&85&68&1651&1&23&80&15&0&26&2\\
    \hline
    Cosine: term-doc max-sqrt &12&145&0&0&93&6&2&1842&6&25&44&15&0&15&1\\
    \hline
    Sum: term-doc mean &6&708&644&28&97&460&170&2522&1&30&190&58&0&43&6 \\
    \hline
    \multicolumn{16}{c}{} \\
    \end{tabular}
    \begin{tabular}{|l|r|r|r|r|r|r|r|r|r|r|r|r|r|r|r|r|r|r|r|r|r|r|r|r|r|r|r|r|r|r|}
    \hline
     Measure &Ic&Em&Co&Fi&Ac&Os&Pr&He&Hy&Ja&Ja&Je&Ja&Mu&Tr &\scriptsize{\textbf{\# in top 20}} \\
    \hline
    Sum: disease matrix &6&0&2&0&3000&1&9&3&84&39&3&1&39&2&59 & \scriptsize{\textbf{20}} \\
    \hline
    Cosine: term-doc mean-sqrt &218&3&13&2&3000&1&9&14&78&84&48&3&84&1&62 & \scriptsize{\textbf{13}} \\
    \hline
    Cosine: term-doc max-sqrt &0&10&0&3&3000&7&46&4&128&115&1&24&115&2&1 & \scriptsize{\textbf{19}} \\
    \hline
    Sum: term-doc mean &136&3&39&16&2636&0&8&50&33&115&53&31&115&121&124 & \scriptsize{\textbf{8}} \\
    \hline
  \end{tabular}
\end{tiny}
\end{table}

Not only having the best average but also the right disease 9 out 13 (BMJ) and 20 out of 30 (Oprha.net) 
in the top 20 out of over 3000 diseases returned from a top 3000 document scores, using the simple sum 
similarity measure on a disease matrix seems to give both best recall and precision. This result is very 
interesting since the document-summed disease matrix was originally made as model for fast tests before 
implementation in the large term document matrix. This could imply that a summation of the document vectors 
for each individual disease seems to enchance the values of information carrying terms with the TF-IDF taking 
care of too common and non-information containing terms. The summation also efficiently eliminates the problem 
of noisy overview articles \ref{Overview}. \\

One of the noteworthy things that can be learned from the bar charts made in this and 
the two previous sections is that there should be a lower bound on the number of documents 
per disease. Acropectorovertebral dysplasia is a premium example that the system needs to 
have a lower bound on the number of medline records that are gathered for each disease. 
This is in order to ensure that the system will be able make a reasonable qualified guess 
on the disease.

\subsection{Clustering of the results}
\fxnote{Needs to be refined}
Currently we are only able to make clusters from disease matrices, we have chosen to make a cluster 
of the top 20 diseases return to see how these lie in relation to each other. Of special interest is 
Cat scratch disease which our system almost always lists correctly. 
%See dendrogram \ref{Dendro_cat_scratch_disease}, it can be seen how ...

%% \begin{figure}[H]
%%         \begin{center}
%%           \includegraphics[width=0.9\textwidth]{barcharts/diseaseMatrix_orphan_hist_NOTnorm_3000_s_cos_sqrt_cos_sum_nn.png}
%%         \end{center}
%%         \caption{Test of non-stemmed mean, median and max using normalization and cosine}
%%         \label{Dendro_cat_scratch_disease}
%% \end{figure}


%% \subsection{The reduced semantic space and keyword extraction}
To be written...

\subsection{On overview article noise and concensus normalization\label{Overview}}

\textbf{Overview articles}\\
Unfortunately there is overview articles that can pollute the search results and if overview articles are 
found in many of the top scoring diseases, it could present a problem. When we run the concensus method as 
described in the \ref{CosineScore}, an overview article would potentially get an unfair high score since it 
gets summed up to 240 times. Though overview articles represents an element of noise, the normalization of 
the vectors in the vector space model should in theory down weight the highly summed documents. We also 
tested to the most common overview article (240 occurrences) --- to see if it could be a problem --- using the 
Orpha.net disease cases among top 3000 (documents). The overview article was present in less than 1 out of 
8 searches which is not a significant amount. The disease matrix on the other hand is less prone to the same 
problem, as it summarizes all information about a disease into one vector. \\

\textbf{Concensus normalization} \\
During a point in the testing of the term document matrices, we got the idea to try and divide each label with 
the number of documents it had been summed over. This could in theory normalize the label in the top score of 
returned results, as labels being over-represented in e.g. the bottom of the top score list would be weighted down. 
However, as this might be a good theoretical idea, it did not quite amount ot anything useful. The results for 
running this on the stemmed term document matrix, using the cosine measures, is: \\
{\small
Mean: [99, 210, 804, 1216, 507, 667, 167, 309, 502, 50, 330, 695, 424] \\
Median: [1036, 989, 1432, 948, 668, 1301, 1315, 1687, 1429, 1233, 1696, 1494, 1322] \\
Max: [1034, 989, 1447, 1084, 635, 1284, 1293, 1687, 1414, 1233, 1696, 1491, 1321] \\
}
It does not take a bar chart to see that these values are pretty off the top 20. The idea might be good enough but it would have to be on a different model or data set than the one we use.
