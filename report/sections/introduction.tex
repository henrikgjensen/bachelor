\chapter{Introduction\label{Introduction}}

Due to the vast amount of information available on the Internet today,
it is near impossible for researchers to have an 'up-to-date'
knowledge on everything but their own specific field. Even that seems
to become more and more difficult as new information is added each day
\cite{CitAddMedLine} \footnote{Take for instance the biomedical
  database MedLine that grows with over half a million citations per
  year}. Therefore it can be necessary to employ tools to help gather,
structure, look for relations or hypothesis within the piles
information. One very popular method of finding new relations in data
is through the use of \textit{text mining}.

\begin{center}
{\small``The whole is greater than the sum of its parts'' $\sim$ Aristotle
(384--322 BC)} 
\end{center}

Text mining refers to the automated search for meaningful patterns in
structured or unstructured text documents stored in very large digital
databases or distributed over the Web. A good example of this is
``Fish oil, Raynaud's syndrome and undiscovered public knowledge'' by
D.R. Swanson \fxnote{REF Don R. Swanson
  -http://www.csd.uwo.ca/courses/CS9626a/papers\_files/Swanson.pdf}. Here
Swanson was referring to published knowledge deeply buried in disjoint
topical domains \footnote{Here 'disjoint' refers to articles written
  by researchers unaware of each others work}. Swanson was one of the
first to propose using text mining on biomedical literature. In 1986,
he found evidence of a relation between the use of fish oil and the
development of Raynaud's syndrome by looking at seemingly unrelated
documents. This was done years before there were actually any
scientific documents supporting this. \\

Over the recent years the use of text mining has grown
tremendously. In the biomedical field, research is divided into highly
specialized sections and subsections often too complex to make room
for interdisciplinary work. For instance, the recent sequencing of the
human genome have introduced a whole new level of detail to genetic
research. It is likely that new discoveries in this area could affect
other areas concerned with health and diseases since genomic
mechanisms play a major role in the various branches of medicine
\cite{survey.biomed.text.cohen.2005}. Text mining is a way of making
these important connections in a world of increasing complexity and
hidden patterns. This way of making new unseen discoveries also
introduces text mining as a major potential aid in the diagnosis of
rare or (to the physician) unknown diseases.\\

This project aims to make it easier and more efficient for physicians
in diagnosing rare diseases. Through the use of text mining,
clustering and machine learning algorithms, it is an attempt to
increase the likelihood of getting the correct article based on a
search of symptoms, environmental and/or human factors. We use a list
of rare diseases and synonyms acquired from rarediseases.info
\cite{Rarediseases}. Based on this, we
extract a series of MedLine records \cite{PubMedFactSheetMedline}
using the Python module Bio.Entrez \cite{EntrezProgUtil} and process
the text for a more optimized search.

\section{Inspiration}

The list of rare diseases counts over 6,000 individual cases and has 5
added to it a week \cite{AboutRareDiseasesOrphanet}. A rare (or
orphan) disease is classified by the Orphanet Encyclopedia
\cite{OrphanetEncyclopedia} being a disease that affects 1 out of 2,000
people in Europe and has severe chronic or terminal outcomes (or less
than 200.000 affected in the USA by standard of
Rarediseases.info \cite{Rarediseases}). Some
of these diseases might not be fatal if treated in time but given the
amount of knowledge your physician would need to carry around to make
a correct diagnosis (or correctly exclude other potentials), this is
not always the case.\\

When being affected by a rare disease, the lack of a correct diagnosis
- or the delay spent going from one specialist to another - will in
often lead to a fatal outcome. When it comes to rare and often
dangerous diseases, the typical physician has little or none prior
experience with similar cases. Therefore it is important that the
diagnosing physician has as much help at hand as possible in this
intrinsic task.\\

In a dialog with Henrik L. Jørgensen, chief physician at Bispeberg
Hospital, we found that though many systems already exist to help
physicians in their diagnosis, there seemed to be a lack of a system
for specifically diagnosing rare diseases. Systems such as PubMed
returns numerous results if the symptoms are slightly non-specific
(more on this in the following chapters). The advantage of a
specialized system for rare diseases would be that the physician,
being in doubt, would have a chance to make a quick symptom look-up
before referring or dismissing the patient. According to Jørgensen,
time is great a problem when treating patients and in the rare event
of a patient being affected by something unknown,he or she is referred
to a specialist.\\

The inspiration of being able to create an efficient support decision
system for diagnosing rare diseases was what drove us to initiate this
project.

\section{Objectives}

\textbf{Aim} \\
Our system will be based on machine learning concepts and will
hopefully add something new to the arena of medical support decision
systems. Testing various techniques to optimize our system, we aim to
design a system that, if successful, also has the generic potential of
being expanded to other domains than that of rare diseases.\\

\textbf{Overall process} \\
The list of rare disease names (along with synonyms and optional
descriptions) will be mined from the Rarediseases.info website. Some
of these diseases will have specialized PubMed search strings that we
mine along with the names. Using these names, predefined search
strings and synonyms, we search PubMed for a maximum of 500
PMID's \footnote{PMID's are unique article identifiers used by PubMed}
per disease, representing MedLine records containing an abstract. We
then download the corresponding records.\\

The intention is to preprocess the data using a vector space model and
various heuristics to optimize the probability of getting a correct
hit. The heuristics are described in the following chapter but
revolves mainly around the Term Frequency - Inverse Document Frequency
(TF-IDF). Since a graphical user interface will not be made for the
prototype system, a correct hit is as an article defining the disease
being among the top 20 returned results from a given query of
symptoms.\\

We will be running tests on three different test cases. The first set
of cases are derived from a subset of tests in the BMJ article
\cite{HangwiTang11102006} relevant to our database\footnote{We only
  run tests on the diseases present in our database} The second set of
cases come from a random select of disease descriptions on the
Orpha.net website \cite{Orphanet}. The third and final set of test
cases come from a blind test provided by Henrik L. Jørgensen
\cite{TheDude}.\\

\textbf{Primary tools} \\ We will be using Python 2.6.2
\cite{PythonLanguage} in this project. For access to PubMed, we use
the python module Bio.Entrez \cite{EntrezProgUtil} while BeautifulSoup
\cite{BS} is used for parsing of HTML/XML combined with Urllib2
\cite{UL2} these are used for crawling the websites. For construction
of the term document matrix, we use the scipy package \cite{SciPy}
which supports sparse matrix structures. And for auxilliary vector
functions we use package \cite{NumPy}.

\section{Roadmap}

Chapter \ref{Background} covers the different the areas that we harvest our
data from and how we intend to model it. It provides a background on
Rarediseases.info, on PubMed and on the MedLine database from where we
get our primary data. It examines the various advantages and
disadvantages of the models and heuristics that we use, how we measure
the similarities used to provide a disease score for queries, and it
looks into the datatypes used to handle the large amounts of
information. Lastly, it looks at an alternative that might be able to
greatly expand the information that we have on each mined disease.\\

Chapter \ref{Methods} deals with the methods that we have used to implement
the first prototype and the following branches of the system. It gives
an overview of the implementation, describing all the main modules and
their relation to each other. It amplifies the flexibility of the data
exchange between the modules and how we have applied several
heuristics/filters to the data and the vector space. The chapter is
rounded of with details of the data and technical conclusions of the
implementation.\\

Chapter \ref{ExperimentsResults} contains all the primary tests and results of the different schemes used to find the most efficient model for looking up rare diseases. The cosine similarity measure and the simple sum similarity measure are tests, measured and put up against each other to see which performs the best (returns most correct diseases in top 20 and has the best average score). This is done on both the classical term document matrix and on a document-summed version called the \textit{thedisease matrix}. Having found the best score, the following sections of this chapter deals with aspects and tests of clusterings, keyword extraction in a reduced semantic space and some of the potential noise in the data set.\\

Chapter \ref{Conclusion}

(Conclusion)\\

Chapter \ref{FutureWorks}

(Future works) \\
