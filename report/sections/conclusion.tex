\chapter{Conclusion}

We choose to use the vector space model because it gives a good base
on which to perform different transformations on the data and performing
similarity scores.

When applying the square root transformation during calculation of
similarity score on the TF-IDF matrix, our results improved, this
suggests that other transformation to the data should be
used. Unfortunately time did not allow us to perform test on other
transformations.

Our system correctly placed 8 out of 13 diseases among top 20 when
using the test cases from BMJ using sum on the disease matrix. When
running on the Orpha.net test cases our system placed 20 out of 29 in
top 20 using sum on disease matrix. The Orpha.net result is closely
follow by using cosine max, with square root transformation on a term
document matrix, this method places 19 out of 29 in top 20.

The blind tests provided by Henrik L. J\o rgensen were also tested
using our system, and the system placed 3 out of 5 within top
20. Though our system contains information about the two diseases not
placed within top 20. We are unable to provide any explanation for
this.

The test results shows that the system correctly places $61.5\%$ of
BMJ test cases, $68\%$ of Orpha.net and $60\%$ of the blind test cases
in top 20.

The elimination of overview article in the database could help reduce
some of the noise found during testing the system. Combined with
outlier detection this might improve the results to some degree.

There is a clear need for more control over the information gather to
describe the rare diseases in the database, this includes a lower
bound on the information gather for a single disease. This is
especially with Acropectorovertebral dysplasia in mind which never
scores below 562 in any test, and in most cases is not even found
among top 3000.

We were unable to find any information about other system designed to
perform automated diagnosis of rare diseases. And in that relation
this system is --- yet --- unique. Hopefully other systems will be
made to assist physicians in diagnosing rare diseases.

When designing and testing a system for diagnosing rare diseases it is
imperative to be able to draw on the knowledge and ideas of domain
specialists. When testing the system against the Orpha.net cases we
choose the symptoms based on a strings from their web page described
by ``characterized by'', a physician might not use the same
description. The blind tests symptoms where received in Danish which
meant that we had to translate it to English, and some information
might have been lost in the translation. A physician have a domain
knowledge and intuition about searching for diseases far exceeding
that of computer scientists designing such a system.

We had a short correspondence with Henrik L. J\o rgensen about the
usefulness of the system. His opinion was that the system could become a
helpful tool for physician, because through their carrier they might
only experience a single rare disease. So in case something did not
quite fit with the diagnosis, they could type in a symptom list and
receive a list of suggested diseases. Which would result in patients
could be referred to the correct treatment.

Considering the test results above this provides evidence that a
system for providing support decision with respects to diagnosing rare
disease have potential.
