\chapter{Conclusion\label{Conclusion}}

We choose to use the vector space model because it gives a good base
on which to perform different transformations of the data and performing
similarity scores.

When applying the square root transformation during calculation of
similarity score on the TF-IDF matrix, our results improved. This
suggests that other transformations to the data should be
used. Unfortunately time did not allow us to perform test on other
transformations.

Using a disease matrix instead of a term-document matrix gave us the best
results on the BMJ ($61.5\%$ correct in top 20) and the Orpha.net ($68\%$ correct in top 20) 
test cases. The blind tests provided by Henrik L. J\o rgensen were also tested
using our system, and the system placed 3 out of 5 within top
20 ($60\%$). Further analysis of the blind tests indicate that protein/gene names
can be a highly noisy factor, as it seemed to be in one of the blind
tests.

The elimination of overview articles in the database could help reduce
some of the noise found during testing the system. Combined with
outlier detection, this could improve the results.
Though a module for outlier detection has been implemented, this was 
not tested.

Latent Semantic Analysis could be a useful tool for reducing the amount 
noisy data by means of dimensionality reduction through SVD. SVD was 
implemented in the system but never fully tested.

There is a clear need for more control over the information gathered to
describe the rare diseases in the database. This includes a lower
bound on the information gather for a single disease. This is
especially with the disease \textit{Acropectorovertebral dysplasia} in mind which never
scores below 562 in any test.

We were unable to find any information about other system designed to
perform automated diagnosis of rare diseases. In this relation
the system build is (yet!) unique. Hopefully other systems will be
made to assist physicians in diagnosing rare diseases.

When designing and testing a system for diagnosing rare diseases it is
imperative to be able to draw on the knowledge and ideas of domain
specialists. When testing the system against the Orpha.net cases, we
choose the symptoms based on a strings from their web page described
by ``characterized by'' - a physician might not use the same
description. The blind tests symptoms where received in Danish which
meant that we had to translate it to English and some information
might have been lost in the --- perhaps non-medicinal ---translation. A physician have a domain
knowledge and intuition about searching for diseases that far exceeding
that of computer scientists designing such a system.

We had a short correspondence with Henrik L. J\o rgensen about the
usefulness of the system. His opinion was that the system could become a
helpful tool for physician because --- through their carrier --- they might
only experience a single rare disease. So in case something did not
quite fit with the diagnosis, they could type in a symptom list and
receive a list of suggested diseases. This could result in more patients
being given the correct treatment.

Considering the test results above, this provides evidence that a
system for providing support decision with respects to diagnosing rare
disease have high potential.
