\section{Introduction}


\subsection{What already exists?}
* Due to the enourmous amount of information available on the internet it is impossible for researchers to have 'up-to-date' knowledge all but their own specific field (even that seems to be impossible in the near future). Therefore it can be necessary to employ tools to help gather, structure, look for relations or hypothesis within information. A good example of this is Swanson DR. Fish oil, Raynaud's syndrome, and undiscovering public knowledge. He found evidence of a relation between the use of fish oil and the development of Raynaud's syndrom

* There is alot of research into automated systems to detect genes and protein-to-protein interaction within research papers using text mining and well annotated training sets (e.g.???). 

\subsection{Medline}
One of largest information resources available on the subject of biomedical articles and cases is the Medline database, containing over 16 million articles <FIXME, insert reference http://en.citizendium.org/wiki/MEDLINE>, dating back as far as 1966 to the present. It is possible to use the http://www.pubmed.gov interface to search and the organization behind Pubmed also provides an interface through a program called Entrez <FIXME, insert reference http://en.wikipedia.org/wiki/Entrez and pubmed page http://www.ncbi.nlm.nih.gov/sites/gquery>, which enables researcher to make automated program that search, download and process the information collected from e.g. Medline.

Our project relies on Entrez to collect the information (abstracts, title, author, source and year). 

\subsection{Rare diseases}
We use http://rarediseases.info.nih.gov <FIXME Insert Henrik (Doctor as reference)> which contains an detailed list of rare diseases. The page includes links for finding resource about rare diseases. Some of the rare diseases listed here have a link to either a crafted pubmed search string or to a precalculated search result. For those that do not have, we have choosen to search for the disease name and synonyms (if any).

