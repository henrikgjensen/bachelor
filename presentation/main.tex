\documentclass[xcolor=table]{beamer}

% Get settings
\usepackage{mybeamer}

% Formalia
\title{ \textbf{Support decision system for diagnosing rare diseases using vector space model and medical text mining}}
\subtitle{{\footnotesize DIKU Bachelorprojekt 2009 -- 2010}\\{\tiny Henrik G. Jensen og Michael Andersen}}
\date{Vinter 2010}

\begin{document}

\begin{frame}
    \titlepage
\end{frame}

\section{Oversigt}

%% \begin{frame}

%%     \frametitle{\ }

%%       \begin{itemize}
%% 	  \item Motivation
%% %%	  \item Strukturering af information
%%           \item Midlet
%% 	  \item Resultater
%% 	  \item Fremtidige forbedringer /  udviddelser ??? 
%%       \end{itemize}

%% \end{frame}

% Begin actual slides
% 1
\section{Diagnosticering af sjældne sygdomme ???}

% 1.1
\subsection*{}

\begin{frame}

  \frametitle{Motivation}

  \visible<1->{
  \begin{block}{Problemet}

    \begin{itemize}
      \item L\ae ger har stor viden, men \ldots
      \item L\ae ger har begr\ae nset tid per patient
      \item Sj\ae ldne sygdomme drukner i informations m\ae ngden.
    \end{itemize}
    
  \end{block}
  }

  \visible<2->{
  \begin{block}{Behovet}

    \begin{itemize}

      \item Der findes mange specifikke systemer, som bestemmer enkelt sygdomme, intet der rammer flere sygdomme.
      \item Giver foreslag til sygdomme, som hurtigt kan kontrolleres.

    \end{itemize}
    
  \end{block}
  }

%%% RODDET!!! %%%
%%   \begin{itemize}
%%     \item <1> Lægen ved ikke alt
%%     \item <1> Manglende tid
%%     \item <1> Mængden af information
%%     \item <1> Globalisering
%%   \end{itemize}
%%%%%%%%%%%%%%%%%%

\end{frame}

\begin{frame}

  \frametitle{Midlet}

  \begin{block}{L\o sningen, systemet}

    \begin{itemize}

      \item Konstruer speciel database
      \item V\ae gtning af termer
      \item Vektor rums modellen
      \item Udregning af score for s\o gning
      \item Forsl\aa\ top 20 sygdomme.
    \end{itemize}

  \end{block}

  \begin{block}{Resultater}

    \begin{itemize}

      \item BMJ test sager
      \item Orpha.net test sager
      \item Blind testen
    \end{itemize}

  \end{block}

  \begin{itemize}
    \item Fremtids muligheder
  \end{itemize}

\end{frame}

\begin{frame}

  \frametitle{Oversigt over systemet}

  Inds\ae t overview diagram over systemet

\end{frame}

\begin{frame}

  \frametitle{Konstruktion af databasen}

  Inds\ae t tegning af interaktion mellem vores system og rarediseases, samt vores system og pubmed, endende med Medline Records.

  (evt. se figur 2.2 side 17)

\end{frame}

\begin{frame}

  \frametitle{V\ae gtning af term}

  Log-transformation af ord antallet:

  \[
  x_{dw}^{log} = \log{(1 + x_{dw})}
  \]

  Der bruges TF-IDF til v\ae gtning af termer, dette er for at fremh\ae ve term som sj\ae lendt optr\ae der og nedv\ae gte ofte forekommende ord:

  \[
  x_{dw}^{tfidf} = x_{dw}^{log} \cdot \log{\frac{D}{\sum_{d\prime = 1}^{D}\delta_{d\prime w}}}
  \]

  Afslutningsvis normaliseres dokument vektoren for at sikre de har ens indflydelse p\aa\ s\o geresultatet.

  \[
  x_{dw}^{norm} = \frac{x_{dw}^{tfidf}}{\sqrt{\sum_{w\prime = 1}^{W} {x_{dw}^{tfidf}}^{2}}}
  \]

\end{frame}

\begin{frame}

  \frametitle{Vector rums modellen}

  Udregning af vinkel mellem en symptom liste og et document:
  
  \[
  \cos \theta_{D_j} = \frac{Q \circ D_j}{|Q| \cdot |D_j|}
  \]
  
  Ovenst\aa ende svarer til
  
  \[
  \textrm{score}_{d} = \frac{1}{|I|}\frac{1}{|x_{d}|} \sum_{i \in I} x_{dw}
  \]
  
  Hvis der p\aa\ forh\aa nd er foretaget normalisering af dokument vektoren:
  
  \[
  \propto  \sum_{i \in I} \widehat{x_{dw}}
  \]
  
\end{frame}

\begin{frame}

  \frametitle{Udregning af score for en sygdom}
  
  \begin{itemize}
    \item Cosine Mean
      \begin{itemize}
        \visible<1->\item $ \textrm{disease}_j = \{\textrm{d1}_{score}, \textrm{d2}_{score}, \textrm{d3}_{score}, \textrm{d4}_{score} \} $
        \visible<2->{\item final score: $\textrm{disease}_j = \textrm{mean}(\{\textrm{d1}_{score}, \textrm{d2}_{score}, \textrm{d3}_{score}, \textrm{d4}_{score} \}) $}
      \end{itemize}
    \item Cosine Median
      \begin{itemize}
        \visible<1->\item $ \textrm{disease}_j = \{\textrm{d1}_{score}, \only<1-2>{\textrm{d2}_{score}} \only<3->{\textbf{d2}_{\mathbf{score}}}, \textrm{d3}_{score}, \textrm{d4}_{score} \} $
      \end{itemize}
    \item Cosine Max
      \begin{itemize}
        \visible<1->\item $ \textrm{disease}_j = \{\textrm{d1}_{score}, \textrm{d2}_{score}, \textrm{d3}_{score}, \only<1-3>{\textrm{d4}_{score}} \only<4->{\textbf{d4}_{\mathbf{score}}} \} $
      \end{itemize}
    \item Sum (P\aa\ disease matrix)
      \begin{itemize}
        \item final score: $\textrm{disease}_j \sum_{i \in I} t_{ji} $
      \end{itemize}
  \end{itemize}

\end{frame}

\section{Resultater}

\subsection*{}

\begin{frame}

  \frametitle{BMJ Resultater}

  Inds\ae t nogle af de gode resultater, evt. nogle af de d\aa rlige.

\end{frame}

\begin{frame}

  \frametitle{Orpha.net Resultater}

  Inds\ae t nogle af de gode resultater, evt. nogle af de d\aa rlige.

\end{frame}

\begin{frame}

  \frametitle{Blind tests Resultater}

  Inds\ae t nogle af de gode resultater, evt. nogle af de d\aa rlige.

\end{frame}

\begin{frame}

  \frametitle{Forkerte forslag}

  Inds\ae t noget om keywords

\end{frame}

\begin{frame}

  \frametitle{Forkerte forslag 2}

  Vi har for lidt information om den enkelte sygdom, den ``drukner'' i symptomerne p\aa\ andre sygdomme.

\end{frame}

\section{Fremtidige muligheder}

\subsection*{}

\begin{frame}

  \frametitle{Udvidelser}

\end{frame}


% 1.2

%% \subsection*{}
%% \begin{frame}

%%   \frametitle{Sjældne sygdomme}

%%   % Evt ledt videre fra 'globalisering'?
%%   \begin{itemize}
%%     \item 
%%     \item
%%   \end{itemize}
  

%% \end{frame}


\end{document}
