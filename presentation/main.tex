\documentclass[xcolor=table]{beamer}

% Get settings
\usepackage{mybeamer}

% Formalia
\title{ \textbf{Support decision system for diagnosing rare diseases using vector space model and medical text mining}}
\subtitle{{\footnotesize DIKU Bachelorprojekt 2009 -- 2010}\\{\tiny Henrik G. Jensen og Michael Andersen}}
\date{Vinter 2010}

\begin{document}

\begin{frame}
    \titlepage
\end{frame}

\section{Oversigt}

\subsection*{}

%% \begin{frame}

%%     \frametitle{\ }

%%       \begin{itemize}
%% 	  \item Motivation
%% %%	  \item Strukturering af information
%%           \item Midlet
%% 	  \item Resultater
%% 	  \item Fremtidige forbedringer /  udviddelser ??? 
%%       \end{itemize}

%% \end{frame}

% Begin actual slides
% 1

\begin{frame}

  \frametitle{Motivation}

  \visible<1->{
  \begin{block}{Problemet}

    \begin{itemize}
      \item Begr\ae nset tid per patient.
      \item Ukendte symptomer / læge i tvivl.
      \item Sj\ae ldne sygdomme drukner i m\ae ngden af information.
    \end{itemize}
    
  \end{block}
  }

  \visible<2->{
  \begin{block}{Behovet}

    \begin{itemize}
    
      \item System specifikt for sjældne sygdomme.
      \item System til hurtige nemme opslag for lægen i tvivl.
      \item System som er up-to-date og informations-rigt.

    \end{itemize}
    
  \end{block}
  }

\end{frame}

\begin{frame}
  
  \only<1>{Hvad gik blind testen ud p\aa ?}
  
  \begin{scriptsize}
    
    \only<2>{

      \hspace*{4mm}

      \begin{tabular}{| p{5cm} | p{5cm} |}
        \hline
        
        Dreng, normal ved f\o dslen bortset fra deformitet af begge
        storet\ae er (de manglede et led). Udvikler sig normalt efterfl\o
        gende. Ved 5 \aa rs alderen der viser knoglev\ae v uden
        malignitetstegn. Kort tid efter biopsien udvikles mere
        knoglev\ae kst, pr\ae cis der hvor man har sk\ae ret. & Boy,
        normal birth, deformity of both big toes (missing joint),
        quick development of bone tumor near spine and osteogenesis at
        biopsy.\\

        \hline
        
        Normally developed boy until age 5, where he progressively
        developed the following symptoms: Talking difficulties, seizures,
        ataxia, adrenal insufficiency and degeneration of visual and
        auditory functions.& Normally developed boy age 5, progessive
        development of talking difficulties, seizures, ataxia, adrenal
        insufficiency and degeneration of visual and auditory functions \\
        
        \hline
        
        A boy age 14 comes to the doctor with yellow, keratotic plaques on
        the skin of his palms and soles going up onto the dorsal
        side. Both hands and feet are affected. He equally had swollen and
        very vulnerable gums since the age of 4 with loss of most of his
        permanent teeth. & Boy age 14, yellow, keratotic plaques on the
        skin of palms and soles going up onto the dorsal side. Both hands
        and feet are affected.\\
        
        \hline
        
      \end{tabular}
    }


    \only<3>{

      \hspace*{4mm}

      \begin{tabular}{| p{5cm} | p{5cm} |}
        
        \hline
        
        16-\aa rig j\o disk dreng har en til to gange om m\aa neden anfald,
        hvor han f\o rst og fremmest skal sove utroligt meget - ca. 18
        timer om dagen. Anfaldene varer ca en uges tid. Han \ae ndrer
        karakter under anfaldene og bliver irritabel og aggressiv, n\aa r
        han v\ae kkes. N\aa r han er v\aa gen i anfaldsperioden spiser han
        helt utroligt store m\ae ngder mad, og hans appetit p\aa\ sex er
        endvidere abnormt stor. & Jewish boy age 16, monthly seizures,
        sleep deficiency, aggressive and irritable when woken, highly
        increased sexual appetite and hunger.\\
        
        \hline
        
        The patient is a male child presenting at birth with numerous
        malformations. He had midfacial retraction with a deep groove
        under the eyes, and hypertelorism. A short nose with a low nasal
        bridge and large low-set ears were noted. He had a wide mouth and
        retrognathia. Hypertrichosis with bright reddish hair and a median
        frontal cutaneous angioma were present. The neck was short with
        redundant skin. Bilateral inguinal hernias, hypospadias with a
        megameatus, and cryptorchidism were noted. & Male child,
        malformations at birth, midfacial retraction with a deep groove
        under the eyes, and hypertelorism, short nose with a low nasal
        bridge and large low-set ears, wide mouth and
        retrognathia. Hypertrichosis with bright reddish hair and a median
        frontal cutaneous angioma, short neck with redundant skin,
        Bilateral inguinal hernias, hypospadias with a megameatus, and
        cryptorchidism \\
        
        \hline
        
      \end{tabular}
    }      
      
      \only<4->{
        \begin{tabular}{| p{5cm} | p{5cm} |}
          
          \hline
          
          16-\aa rig j\o disk dreng har en til to gange om m\aa neden anfald,
          hvor han f\o rst og fremmest skal sove utroligt meget - ca. 18
          timer om dagen. Anfaldene varer ca en uges tid. Han \ae ndrer
          karakter under anfaldene og bliver irritabel og aggressiv, n\aa r
          han v\ae kkes. N\aa r han er v\aa gen i anfaldsperioden spiser han
          helt utroligt store m\ae ngder mad, og hans appetit p\aa\ sex er
          endvidere abnormt stor. & \only<5->\textbf{Jewish} boy age 16, monthly \only<5->\textbf{seizures},
          \only<5->\textbf{sleep deficiency}, \only<5->\textbf{aggressive} and \only<5->\textbf{irritable} when woken, highly
          \only<5->\textbf{increased sexual appetite} and \only<5->\textbf{hunger}.\\
          
          \hline
          
        \end{tabular}
        
      }

  \end{scriptsize}

  \hspace*{4mm}

  \visible<6->{Hvad er det for en sygdom?}

  \hspace*{4mm}

  \visible<7->{
    \begin{center}
      \textbf{Kleine Levin syndrome}
    \end{center}
  }

\end{frame}

\begin{frame}

  \frametitle{Midlet}

  \begin{block}{L\o sningen, systemet}

    \begin{itemize}

      \item Konstruer speciel database.
      \item Vector space model.
      \item V\ae gtning af termer.
      \item Udregning af sygdoms-score.
      \item Forsl\aa\ top 20 sygdomme.

    \end{itemize}

  \end{block}

\end{frame}

\section{Systemet}

% 1.1
\subsection*{}

\begin{frame}

  \frametitle{Oversigt over systemet:
    \only<1>{Crawler}\only<2>{MedLine}\only<3>{TermDoc}\only<4>{TermDoc
      / DiseaseMatrix}\only<5>{TF-IDF}\only<6>{Heuristikker}\only<7>{Outlier detection}\only<8>{LSA}\only<9>{Yderligere informationskilder}}

  \begin{center}
    \only<1>{\includegraphics[width=1.0\textwidth]{diagram/system_overview1}}
    \only<2>{\includegraphics[width=1.0\textwidth]{diagram/system_overview2}}
    \only<3>{\includegraphics[width=1.0\textwidth]{diagram/system_overview3}}
    \only<4>{\includegraphics[width=1.0\textwidth]{diagram/system_overview4}}
    \only<5>{\includegraphics[width=1.0\textwidth]{diagram/system_overview5}}
    \only<6>{\includegraphics[width=1.0\textwidth]{diagram/system_overview6}}
    \only<7>{\includegraphics[width=1.0\textwidth]{diagram/system_overview7}}
    \only<8>{\includegraphics[width=1.0\textwidth]{diagram/system_overview8}}
    \only<9>{\includegraphics[width=1.0\textwidth]{diagram/system_overview9}}    
  \end{center}

\end{frame}

%% \begin{frame}

%%   \frametitle{Konstruktion af databasen}

%%   Inds\ae t tegning af interaktion mellem vores system og rarediseases, samt vores system og pubmed, endende med Medline Records.

%%   (evt. se figur 2.2 side 17)

%% \end{frame}

\begin{frame}

  \frametitle{Vector space model - Term doc matrix}

  \begin{center}
    \only<1>{\includegraphics[width=0.65\textwidth]{diagram/termdoc1}}
    \only<2>{\includegraphics[width=0.65\textwidth]{diagram/termdoc2}}
  \end{center}

  $d_j = (x_{j1}, \ldots, x_{jm})^t$ er en transponeret s\o jle vektor og $t_i = (x_{1i}, \ldots, x_{ni})$ er en term vektor.

\end{frame}

\begin{frame}

  \frametitle{V\ae gtning af term}

  Log-transformation af ord antallet:

  \[
  x_{dw}^{log} = \log{(1 + x_{dw})}
  \]

  Der bruges TF-IDF til v\ae gtning af termer. Dette er for at fremh\ae ve term som sj\ae ldent optr\ae der og nedv\ae gte ofte forekommende ord:

  \[
  x_{dw}^{tfidf} = x_{dw}^{log} \cdot \log{\frac{D}{\sum_{d\prime = 1}^{D}\delta_{d\prime w}}}
  \]

  Afslutningsvis normaliseres dokument vektoren for at sikre de har ens indflydelse p\aa\ s\o geresultatet:

  \[
  x_{dw}^{norm} = \frac{x_{dw}^{tfidf}}{\sqrt{\sum_{w\prime = 1}^{W} {x_{dw}^{tfidf}}^{2}}}
  \]

\end{frame}

\begin{frame}

  \frametitle{Vector space model - Cosine score}

  Udregning af vinkel mellem en søgevektor og et dokument:
  
  \[
  \cos \theta_{d} = \frac{\overrightarrow{q} \circ \overrightarrow{d}}{|\overrightarrow{q}| \cdot |\overrightarrow{d}|}
  \]
  
  Ovenst\aa ende svarer til:
  
  \[
  S_{d} = \frac{1}{|\overrightarrow{q}|}\frac{1}{|\overrightarrow{d}|} \sum_{w|w \in \overrightarrow{q}} x_{dw}
  \]
  
  Hvis der p\aa\ forh\aa nd er foretaget normalisering af dokument vektoren:
  
  \[
  \propto  \sum_{w|w \in \overrightarrow{q}} \widehat{x_{dw}}
  \]
  
\end{frame}

\begin{frame}

  \frametitle{Udregning af score for en sygdom}
  Score for sygdom j:
  $\textrm{disease}_j = \{\textrm{S}_{d_{1j}}, \ldots, 
  \textrm{S}_{d_{(n-1)j}}, \textrm{S}_{d_nj} \}$, hvor $d_i$ er dokument $i$.

  \begin{itemize}
    \item Cosine Median
      \begin{itemize}
        \item $ \textrm{disease}_j =
        \{\textrm{S}_{1j}, \only<1>{\textrm{S}_{2j}}
        \only<2->{\textbf{S}_{\mathbf{2j}}}, \textrm{S}_{3j},
        \textrm{S}_{4j} \}$
      \end{itemize}
    \item Cosine Max
      \begin{itemize}
        \item $ \textrm{disease}_j =
        \{\textrm{S}_{1j}, \textrm{S}_{2j},
        \textrm{S}_{3j}, \only<1-2>{\textrm{S}_{4j}}
        \only<3->{\textbf{S}_{\mathbf{4j}}} \}$
      \end{itemize}
    \item Cosine / Sum normaliseret
      \begin{itemize}
        \item \[ \textrm{disease}_j = \sum_{d | d \in \textrm{disease}_j} \sum_{w|w \in \overrightarrow{q}} \widehat{x_{dw}} \]
      \end{itemize}
    \item Sum unormaliseret (P\aa\ disease matrix)
      \begin{itemize}
        \item \[ \textrm{disease}_j = \sum_{w|w \in \overrightarrow{q}} \widetilde{x_{dw}} \]
      \end{itemize}
  \end{itemize}

\end{frame}

\section{Resultater}

\subsection*{}

\begin{frame}

  \frametitle{Diagnosticering af Sygdomme}

  \begin{itemize}
    \visible<1->\item Hvordan testes systemet? Eksempler:
  \end{itemize}

  \visible<2->{
  \begin{tabular}{|l|p{5cm}|}
    \hline
    Infective endocarditis & Acute, aortic, regurgitation, depression,
    abscess \\
    \hline
  \end{tabular}
  }
  
  \hspace*{4mm}

  \visible<3->{
  \begin{tabular}{|l|p{5cm}|}
    \hline
    Cholera \quad & severe diarrhea and vomiting \\
    \hline
  \end{tabular}
  }

  \hspace*{4mm}

  \visible<4->{
  \begin{tabular}{|l|p{5cm}|}
    \hline
    Kleine Levin Syndrome & Jewish boy age 16, monthly seizures, sleep
    deficiency, aggressive and irritable when woken, highly increased
    sexual appetite and hunger \\
    \hline
  \end{tabular}
  }


\end{frame}

\begin{frame}

  \frametitle{BMJ Resultater}

  \begin{center}
    \includegraphics[width=1.0\textwidth]{diagram/termDoc_bmj_hist_3000_sum_dm_mea_cos_sqrt_td_max_cos_sqrt_td_mea_sum_td}
  \end{center}

\end{frame}

\begin{frame}

  \frametitle{Orpha.net Resultater}

  \begin{center}
    \includegraphics[width=1.0\textwidth]{diagram/termDoc_orphan_hist_3000_sum_dm_mea_cos_sqrt_td_max_cos_sqrt_td_mea_sum_nn_td}
  \end{center}

\end{frame}

\begin{frame}

  \frametitle{Blind tests Resultater}

  \begin{center}
    \includegraphics[width=1.0\textwidth]{diagram/blind_test}
  \end{center}

\end{frame}

\begin{frame}

  \frametitle{Forkerte forslag}

  Hvorfor rammer vi ikke den korrekte sygdom?

  \begin{table}[H]
    \begin{tiny}
      \begin{center}
        \begin{tabular}{|c|c|}
          \hline
          \scriptsize{\textbf{Adrenoleukodystrophy}} & \scriptsize{\textbf{Kleine Levin}} \\
          \hline
          pex26 &lein  \\
          \hline
          pex1 &megaphagia \\
          \hline
          pex13 &hypersomnia\\
          \hline
          pex5 &levin\\
          \hline
          nald &hypersomn\\
          \hline
          pts1 &kl\\
          \hline
          ldpex5 &hypersomniac\\
          \hline
          pex12 &hypersexu \\
          \hline
          pex5p &hypersomnol\\
          \hline
          peroxin&asmp \\
          \hline
          pex10 &narcolepsi\\
          \hline
          zs &hypocretin\\
          \hline
          pbd &mslt \\
          \hline
          pex14 &hyperphagia\\
          \hline
          g843d &pickwickian \\
          \hline
          pex12p &somnol\\
          \hline
          pex1p &ceretec\\
          \hline
          ird &parasomnia\\
          \hline
          pts1r &rem\\
          \hline
          pts2 &smp\\
          \hline
        \end{tabular}
      \end{center}
    \end{tiny}
  \end{table}
  
\end{frame}

\begin{frame}

  \frametitle{Forkerte forslag 2}
  \begin{itemize}
    \item\visible<1->{Sygdommen \emph{Acropectorovertebral dysplasia}'s information er baseret p\aa\ en MedLine Record.}

    \item\visible<2->{Der er brug for mere diversitet i termerne / flere synonymer for symptomer.}

    \item\visible<3->{Informationen er baseret p\aa\ h\o jest 500 MedLine Records.}
  \end{itemize}

\end{frame}

\begin{frame}

  \frametitle{Clustering af resultater}

  \begin{center}
     \only<1>{\includegraphics[width=0.90\textwidth]{diagram/cluster_cat_scratch.jpg}}
     \only<2>{\includegraphics[width=0.90\textwidth]{diagram/cluster_cat_scratch2.jpg}}
  \end{center}

\end{frame}

\section{Fremtidige muligheder}

\subsection*{}

\begin{frame}

  \frametitle{Udvidelser}

  \begin{itemize}
    \item Statistisk model.
    \item Tilf\o jelse af andre heuristikker.
    \item Keyword extraction fra sygdomme.
    \item Udvidet mulighed for s\o gning, f.eks. sammenligning af s\o
      gninger og termer ved brug af boolske operatorer.
  \end{itemize}

\end{frame}

\begin{frame}

  \frametitle{Outlier detection}

  Afstandsmatrix hvor cosine er brugt som afstandsm\aa l. Sidste s\o
  jle beskriver Mean Distance To Others.

%  \begin{multicols}{2}
  \begin{center}
    \begin{tabular}{|l|r|r|r|r|r|}
      \hline
      DM    & d$_1$ & d$_2$ & d$_3$ & d$_4$ & MDTO \\
      \hline
      d$_1$ &   1.0 &   0.3 &   0.7 &   0.8 & 0.7 \\
      \hline
      d$_2$ &   0.3 &   1.0 &   0.5 &   0.6 & 0.6 \\
      \hline
      d$_3$ &   0.7 &   0.5 &   1.0 &   0.3 & 0.625 \\
      \hline
      d$_4$ &   0.8 &   0.6 &   0.3 &   1.0 & 0.675 \\
      \hline
    \end{tabular}
  \end{center}

%  \end{multicols}

  Andre metoder, f.eks. afstand fra middel vektor, KNN
  clustering.

\end{frame}

\section{Konklusion}

\begin{frame}

  \frametitle{Konklusion}

  \begin{itemize}
    \visible<2->{\item Rammer omkring 60\%}
    \visible<3->{\item Andre transformation b\o r afpr\o ves}
    \visible<4->{\item Minimum gr\ae nse for information per sygdom}
    \visible<5->{\item St\o j reduktion n\o dvendigt}
    \visible<6->{\item Medicinsk dom\ae ne viden n\o dvendigt}
  \end{itemize}

  \hspace*{1cm}

%  \begin{center}
  \visible<6>{\textbf{Tak!}}
%  \end{center}

\end{frame}

\end{document}
