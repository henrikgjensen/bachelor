\documentclass[a4paper, 10pt, english, final]{report}
%\usepackage[utf8]{inputenc}
%\usepackage{a4wide}

%e\usepackage[draft]{fixme}



%%%%% TYPE SETTING %%%%%
\usepackage[T1]{fontenc}
%\usepackage{garamond}
%\renewcommand{\rmdefault}{ugm}

\usepackage[british]{babel}
\usepackage{verbatim}

\usepackage{palatino}              % font : garamond
\linespread{1.05}                  % Palatino needs more leading (space between lines)
\renewcommand{\ttdefault}{cmtt}    % alternative monospace font


\usepackage{a4wide}
\usepackage[left=1cm,top=1cm,right=1cm,nohead,nofoot]{geometry}
% How to change margins...
% use voffset for vertical
% use hoffset for horisontal
%% \newenvironment{changemargin}[1]{
%%   \begin{list}{}{
%%       \setlength{\hoffset}{#1}
%%     }
%%   \item[]}{\end{list}}

%%%%%%%%%%%%%%%%%%%%%%%%%%%%%%%%%%%%%%%%%%%%%%%%%%%%%%%%%%%%%%
% Indhold
%%%%%%%%%%%%%%%%%%%%%%%%%%%%%%%%%%%%%%%%%%%%%%%%%%%%%%%%%%%%%%
\begin{document}

%\setlength{\hoffset}{-3cm}

\begin{itemize}

  \item Systemet
    \begin{itemize}
      \item Oversigt
        \begin{itemize}
          \item Data fra raredisease list, crawler til SIR som henter MedLine Records (Stopword removal)
          \item Konstruktion af Term-Doc i stemmed og ikke stemmed udgave
          \item Konstruktion af Disease matrix i stemmed og ikke stemmed udgave
          \item TF-IDF transformation af indgangene i Term-Doc og Disease matrix.
          \item Heuristikkerne
          \item Hvad er der nu af muligheder? Outlier detection
          \item Hvad mere? LSA, lav en Term-Doc baseret p\aa\ det semantiske rum.
          \item Kan der g\o res mere? Tilf\o jelse af flere information kilder.
        \end{itemize}
      \item Vector space model - \textbf{Term-Doc}, hvad er det?
        \begin{itemize}
          \item En matrix, hvor en r\ae kke er et dokument og s\o jlerne udg\o r de forskellige termer.
        \end{itemize}
      \item TF-IDF, v\ae gtning af termer
        \begin{itemize}
          \item Der foretages f\o rst $\log$ transformation af term frekvensen
          \item Der ganges nu den inverse dokument frekvensens p\aa\ .
          \item Dokument vektoren normaliseres nu.
        \end{itemize}
      \item Vector space model - \textbf{Cosine score}
        \begin{itemize}
          \item Standard cosine mellem to vektorer
          \item Det svarer ogs\aa\ til \ldots
          \item Allerede normaliseret dokument vektor?, proportionalt
            med, ser bort fra $\frac{1}{I}$ fungere som en skalar.
        \end{itemize}
      \item Scor en sygdom
        \begin{itemize}
          \item Cosine 'mean', it not really a mean value, it is normalized sum, for real mean see table 4.6
          \item Cosine median, v\ae lger median score i listen.
          \item Cosine max, v\ae lger max score i listen.
          \item Sum, summer dokument v\ae rdierne unormaliseret.
        \end{itemize}
      \item BMJ Cases
        \begin{itemize}
          \item \textbf{Sum-disease matrix} 8 / 13 \textbf{Cosine-mean
            sqrt term doc} 6 / 13 \textbf{Cosine max term doc} 4 / 13
            \textbf{Cosine mean term doc} 4 / 13
          \item Se figur 4.15 side 62
        \end{itemize}
      \item Orpha.net
        \begin{itemize}
          \item \textbf{Sum-disease matrix} 20 / 29
            \textbf{Cosine-mean sqrt term doc} 13 / 20 \textbf{Cosine
              max term doc} 19 / 20 \textbf{Cosine mean term doc} 8 /
            29.
          \item Se figur 4.16 side 63
        \end{itemize}
      \item Blind test
        \begin{itemize}
          \item \textbf{Sum-disease matrix} 3 / 5.
          \item Har kun testet sum med disease matrix, da vi kun
            leverede et svar tilbage til Henrik L. J\o rgensen.
          \item Se figur 4.19, side 66.
        \end{itemize}
      \item Forkerte forslag
        \begin{itemize}
          \item Protein / Gene termer har f\aa et for meget v\ae gt.
          \item Hvorfor
        \end{itemize}
      \item Forkerte forslag 2
        \begin{itemize}
          \item Visse sygdomme er baseret p\aa\ information fra
            f\aa\ MedLine Records.
          \item Mangler diversitet i informationen, mangler synonymer
            for symptomerne.
          \item Informationen fra f\o rste 500 MedLine records. \textbf{Alt}:
            v\ae lg ligefordelt.
        \end{itemize}
      \item Clustering
        \begin{itemize}
          \item Clustering er baseret p\aa\ hvad systemet giver tilbage, alts\aa\ hvad det placerer i top 20.
          \item Cluster sygdommene mod hinanden, p\aa\ trods af at de
            ligger f.eks. nummer 1 og nummer 19 godt minde om
            hinanden, top 20, baseret p\aa\ f\aa\ symptomer,
            informationen \emph{dybere} end 8-10 termer, query vektoren.
        \end{itemize}
      \item Udvidelser
        \begin{itemize}
          \item Statistisk model, prior probability for sygdomme,
            probability bestemt efter lokation, k\o n, race osv.
          \item Tilf\o jelse af andre heuristikker, andre
            transformationer af term frekvensen.
          \item Udtr\ae k af n\o gle ord for sygdomme evt. ved at
            bruge LSA
          \item Sammenligninger af s\o gninger, boolske operatorer til
            s\o gning.
          \item Outlier detection, Vi bruger distance matrix. \textbf{Alt}:
            udregne en middelvektor, kan bruge K-means clustering (Random)
        \end{itemize}
    \end{itemize}

\end{itemize}

Kleine-Levin syndrome (KLS) is a rare disorder with symptoms that
include periodic hypersomnia, cognitive and behavioural
disturbances. Large series of patients are lacking. In order to report
on various KLS symptoms, identify risk factors and analyse treatment
response, we performed a systematic review of 195 articles, written in
English and non-English languages, which are available on Medline
dating from 1962 to 2004. Doubtful or duplicate cases, case series
without individual details and reviews (n = 56 articles) were
excluded. In addition, the details of 186 patients from 139 articles
were compiled. Primary KLS cases (n = 168) were found mostly in men
(68\%) and occurred sporadically worldwide. The median age of onset
was 15 years (range 4-82 years, 81\% during the second decade) and the
syndrome lasted 8 years, with seven episodes of 10 days, recurring
every 3.5 months (median values) with the disease lasting longer in
women and in patients with less frequent episodes during the first
year. It was precipitated most frequently by infections (38.2\%), head
trauma (9\%), or alcohol consumption (5.4\%). Common symptoms were
hypersomnia (100\%), cognitive changes (96\%, including a specific
feeling of derealization), eating disturbances (80\%), hypersexuality
(43\%), compulsions (29\%), and depressed mood (48\%). In 75 treated
patients (213 trials), somnolence decreased using stimulants (mainly
amphetamines) in 40\% of cases, while neuroleptics and antidepressants
were of poor benefit. Only lithium (but not carbamazepine or other
antiepileptics) had a higher reported response rate (41\%) for
stopping relapses when compared to medical abstention
(19\%). Secondary KLS (n = 18) patients were older and had more
frequent and longer episodes, but had clinical symptoms and treatment
responses similar to primary cases. In conclusion, KLS is a unique
disease which may be more severe in female and secondary cases.
Publication Type: Review

\end{document} 
